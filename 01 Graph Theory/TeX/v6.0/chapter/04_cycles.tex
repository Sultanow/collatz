\chapter{Cycles in the Collatz Graph}
\label{ch:cycles}

\section{A remark about cycles}
\label{sec:cycles}
In graph theory, a path of length $n\geq 1$ that starts and ends at the same vertex is called a circuit. A circuit, in which no vertex is repeated with the sole exception that the initial vertex is the terminal vertex, is called a cycle. A cycle of length $n$ is referred to as an $n$-cycle. For these definitions, we rely on \cite[p.~599]{Ref_Rosen}, \cite[p.~35]{Ref_Benjamin_Chartrand_Zhang} and \cite[p.~445]{Ref_Chartrand_Zhang}. Furthermore, we call a cycle originating from the root a trivial cycle.

\begin{remark}
In order for the cycles to become graphically visible, we now require that in a graph $H$ two vertices $v_1$ and $v_2$ are one and the same if the label of both nodes are identical: $l_{V(H)}(v_1)=l_{V(H)}(v_2)\rightarrow v_1=v_2$. As a consequence, there is no guarantee that the graph precisely refers to the algebraic structure of a free monoid anymore. A free monoid requires that each of its elements can be written in one and only one way.
\end{remark}

When different nodes collapse on one, the graph is no longer necessarily a tree. Let us point to the monoid $S^\ast$, which we introduced in section \ref{sec:groups_graphs}. Take for example four of its elements, the empty string $e$, the strings $qqr$, $qqrqqr$, and $qqrqqrqqr$. These elements lie as well within the subset $U\subset T\subset S^\ast$, and they are represented by nodes of the tree $H_U$ that all have the same label $1=ev_{S^\ast}(qqr,1)=ev_{S^\ast}(qqrqqr,1)=ev_{S^\ast}(qqrqqrqqr,1)$. These nodes are one and the same, the root of $H_U$. Visually, then in $H_U$ a directed edge goes from the vertex labeled with $4$ back to the root node. Analogically, in $H_{C,3}$ a loop connects the root to itself, since due to the path contraction even labeled nodes do not exist in $H_{C,3}$. The  aforementioned example reflects the trivial cycle of the Collatz sequence.

Figure~\ref{fig:5} depicts a section of $H_{C,5}$, which includes the $3$-cycle $43,17,27$. Because of the two non-trivial cycles $(43,17,27)$ and $(83,33,13)$, in $H_{C,5}$ there does not exist a path between the root and the vertex $43$ and between the root and the vertex $83$. Hence, $H_{C,5}$ is said to be a disconnected graph. Generally, a graph is called a disconnected graph if it is impossible to walk (along its edges) from any vertex to any other \cite[pp.~46-47]{Ref_Benjamin_Chartrand_Zhang}.

\begin{figure}
	\includegraphics[width=1.00\textwidth]{figures/h_c5a.png}
	\caption{Section of $H_{C,5}$ including the $3$-cycle $43,17,27$}
	\label{fig:5}
\end{figure}

The following considerations focus on non-trivial cycles, and therefore on cycles that do not originate from the root, but cause the graph to be a disconnected graph. Utilizing the example of the graph $H_{C,5}$ we are able to deduct from the cycle $(43,17,27)$ the simple and self-evident equality $\textit{left-child}^3(43)=43$:
\begin{equation*}
\begin{array}{l}
\textit{left-child}(43)=\frac{1}{5}*\left(43*2^1-1\right)=17
\\[\medskipamount]
\textit{left-child}(17)=\frac{1}{5}*\left(17*2^3-1\right)=27
\\[\medskipamount]
\textit{left-child}(27)=\frac{1}{5}*\left(27*2^3-1\right)=43
\end{array}
\end{equation*}

Obviously, the authors note, it would be interesting to find out what circumstances enable a graph to have non-trivial cycles, whether it be the $5x+1$ variant, the $7x+1$ variant of $H_C$ or any variant $H_{C,k}$ with $k\geq 1$.

\section{\texorpdfstring{Which variants of $H_C$ have non-trivial cycles?}{Which variants of HC have non-trivial cycles?}}
\label{sec:non_trivial_cycles}
The generalization of the relationship between successive nodes, given by equation~\ref{eq:generalized_reachability} leads to the condition for an existence of an $n$-cycle in any $kx+1$ variant of $H_C$, which looks analogous to the condition given by equation~\ref{eq:func_cycle} that specifies $H_{C,3}$ has a cycle:
\begin{equation}
\label{eq:generalized_cycle}
2^\alpha=\prod_{i=1}^{n}\left(k+\frac{1}{v_i}\right)
\end{equation}

The natural number $\alpha$ is the sum of edges that have been contracted between the vertices $v_i$ forming the cycle, in other words $\alpha$ is the number of divisions by $2$ within the sequence. The natural number $n$ is the cycle length and $k$ obviously specifies the variant of $H_C$. Since between each vertex at least one edge has been contracted (at least one division by $2$ took place), we know that our exponent alpha is greater than or equal to the sequence length:
\begin{equation}
\label{eq:n_alpha}
\alpha\ge n
\end{equation}

In their 2020 publication Koch et al. \cite{Ref_Koch_2020} provide a list of cycles for different values of $k$, identified with a linear search performed by a Python script \cite{Ref_Koch_Github}. Table~\ref{table:known_cycles} lists all these discovered cycles (refer to \cite{Ref_Koch_2020} for details on the discovery procedure and search intervals). Note that the cycles in table~\ref{table:known_cycles} are written in reverse order, i.e. in the order which corresponds to the Collatz sequence. To obtain the cycles in terms of graph theory referring to the graph $H_C$, read them from right to left.

\begin{table}[H]
	\centering
	\begin{tabular}{|L|R|R|C|}
		\hline
		\thead{\boldsymbol{k}} &
		\thead{\textbf{cycle}} &
		\thead{\boldsymbol{\alpha}} &
		\thead{\textbf{non-trivial}} \\
		\hline
		1 &
		1 &
		1 &
		\\
		\hline
		3 &
		1 &
		2 &
		\\
		\hline
		5 &
		1,3 &
		5 &
		\\
		\hline
		5 &
		13,33,83 &
		7 &
		\checkmark \\
		\hline
		5 &
		27,17,43 &
		7 &
		\checkmark \\
		\hline
		7 &
		1 &
		3 &
		\\
		\hline
		15 &
		1 &
		4 &
		\\
		\hline
		31 &
		1 &
		5 &
		\\
		\hline
		63 &
		1 &
		6 &
		\\
		\hline
		127 &
		1 &
		7 &
		\\
		\hline
		181 &
		27,611 &
		15 &
		\checkmark \\
		\hline
		181 &
		35,99 &
		15 &
		\checkmark \\
		\hline
		255 &
		1 &
		8 &
		\\
		\hline
		511 &
		1 &
		9 &
		\\
		\hline
	\end{tabular}
	\caption{Known $n$-cycles in $kx+1$ variants of $H_C$ for $k\leq1000$, $n\leq 100$}
	\label{table:known_cycles}
\end{table}

Based on the results shown in table~\ref{table:known_cycles} we state the following theorem~\ref{theo:2} that renders more precisely the prerequisite for cycles that may occur in any variants of $H_C$.

\begin{theorem}
	\label{theo:2}
	An $n$-cycle can only exist in a graph $H_{C,k}$, if the following equation holds:
	\begin{equation*}
	2^{\bar\alpha}=2^{\lfloor n\log_2k\rfloor+1}=\prod_{i=1}^{n}\left(k+\frac{1}{v_i}\right)
	\end{equation*}
\end{theorem}

The statement behind theorem~\ref{theo:2} consists in the claim that, in order for an $n$-cycle to occur, the exponent $\alpha$ has to be $\bar\alpha=\lfloor n\log_2k\rfloor+1$. This statement is true if the following general condition for the validity of the cycle-alpha's upper limit always holds (see \cite{Ref_Koch_2020}):
\begin{equation}
\label{eq:condition_max}
n\log_2k-\lfloor n\log_2k\rfloor<2-\log_2\left(\prod_{i=1}^{n}\left(1+\frac{1}{kv_{i}}\right)\right)
\end{equation}

A product $\prod(1+a_n)$ with positive terms $a_n$ is convergent if the series $\sum a_n$ converges, see Knopp \cite[p.~220]{Ref_Knopp}. A similar statement provides Murphy \cite{Ref_Murphy}, who write the factors in the form $c_n=1+a_n$ and explains that if $\prod c_n$ is convergent then $c_n\rightarrow1$ and therefore if $\prod (1+a_n)$ is convergent then $a_n\rightarrow0$. Thus, to verify whether the product in condition~\ref{eq:condition_max} is converging towards a limiting value, it is sufficient to examine the following sum:
\begin{equation*}
\sum_{i=1}^{n}\frac{1}{kv_{i}}
\end{equation*}

The sum of reciprocal vertices depending only from $v_1$ is given in appendix~\ref{appx:sum_reciprocal_vertices}.

\section{Cycles and the product in the condition for cycle-alpha's upper limit}
Let us start with the following product equality, which will give us insights into the relationship between cycles and the product in the condition for alpha's upper limit. The variables $V_1,\ldots,V_m$ and $W_1,\ldots,W_n$ are all odd positive integers:
\begin{equation}
	\label{eq:product_equality}
	(V_1+1)\cdots(V_m+1)\cdot W_1\cdots W_n=V_1\cdots V_m\cdot(W_1+1)\cdots(W_n+1)
\end{equation}

Every natural odd number $V$ can be expressed in the form of $V=v\cdot2^{\alpha}-1$ whereby $v$ is an positive odd integer and $\alpha>0$ is any natural number. This allows us to perform the following substitution (we use $\alpha_V$ for denoting the divisions by two between successive nodes $v_i$ and $\alpha_W$ for divisions by two between successive nodes $w_i$):

\begin{equation}
	\label{eq:product_equality_substitution}
	\begin{array}{lll}
		V_1&=v_22^{\alpha_{V,1}}-1&=kv_1\\
		V_2&=v_32^{\alpha_{V,2}}-1&=kv_2\\
		\vdots&\vdots&\vdots\\
		V_{m-1}&=v_m2^{\alpha_{V,m-1}}-1&=kv_{m-1}\\
		V_m&=v_12^{\alpha_{V,m}}-1&=kv_m
	\end{array}\qquad
	\begin{array}{lll}
		W_1&=w_22^{\alpha_{W,1}}-1&=kw_1\\
		W_2&=w_32^{\alpha_{W,2}}-1&=kw_2\\
		\vdots&\vdots&\vdots\\
		W_{n-1}&=w_n2^{\alpha_{W,n-1}}-1&=kw_{n-1}\\
		W_n&=w_12^{\alpha_{W,n}}-1&=kw_n
	\end{array}
\end{equation}

The substitution rotating from $v_2=(kv_1+1)\cdot2^{-\alpha_{V,1}}$ to $v_m=(kv_{m-1}+1)\cdot2^{-\alpha_{V,m-1}}$ and finally back to $v_1=(kv_m+1)\cdot2^{-\alpha_{V,m}}$ describes a cycle. The result of these substitutions into equation~\ref{eq:product_equality} is the following equality:
\begin{flalign*}
	v_22^{\alpha_{V,1}}\cdots v_m2^{\alpha_{V,m-1}}v_12^{\alpha_{V,m}}\cdot W_1\cdots W_n&=V_1\cdots V_m\cdot w_22^{\alpha_{W,1}}\cdots w_n2^{\alpha_{W,n-1}}w_12^{\alpha_{W,n}}\\
	v_22^{\alpha_{V,1}}\cdots v_m2^{\alpha_{V,m-1}}v_12^{\alpha_{V,m}}\cdot kw_1\cdots kw_n&=kv_1\cdots kv_m\cdot w_22^{\alpha_{W,1}}\cdots w_n2^{\alpha_{W,n-1}}w_12^{\alpha_{W,n}}
\end{flalign*}

The trivial case where $n=m$ and the sum of exponents are equal $\sum_{i=1}^{m}\alpha_{V,i}=\sum_{i=1}^{n}\alpha_{W,i}$ simplifies the product equality as follows:
\begin{flalign*}
	(V_1+1)\cdots(V_n+1)\cdot W_1\cdots W_n&=V_1\cdots V_n\cdot(W_1+1)\cdots(W_n+1)\\
	v_1\cdots v_n\cdot\cancel{2^{\alpha_{V,1}+\ldots+\alpha_{V,n}}}\cdot W_1\cdots W_n&=V_1\cdots V_m\cdot w_1\cdots w_n\cdot\cancel{2^{\alpha_{W,1}+\ldots+\alpha_{W,n}}}
\end{flalign*}
This equality becomes immediatly true if $V_1\cdots V_n=W_1\cdots W_n$ which is the less spectacular case. The more interesting case arises from setting $V_i=kv_i$ and $W_i=kw_i$ as given by substitution~\ref{eq:product_equality_substitution} wich turns the product equality into an always true statement as well:
\begin{flalign*}
	v_1\cdots v_n\cdot W_1\cdots W_n&=V_1\cdots V_m\cdot w_1\cdots w_n\\
	v_1\cdots v_n\cdot k^n\cdot w_1\cdots w_n&=k^n\cdot v_1\cdots v_n\cdot w_1\cdots w_n
\end{flalign*}

\begin{example}
	The following exemplarly product equality fullfills equation~\ref{eq:product_equality}, whereby $V_1=65$, $V_2=165$, $V_3=415$ and $W_1=135$, $W_2=85$, $W_3=215$:
	\[
	(65+1)(165+1)(415+1)\cdot135\cdot85\cdot215=65\cdot165\cdot415\cdot(135+1)(85+1)(215+1)
	\]
	We perform the following substitutions:
	\[
	\arraycolsep=0.2em\begin{array}{ll}
		V_1=65&=v_22^{\alpha_{V,1}}-1=33\cdot2^1-1=5v_1\\
		V_2=165&=v_32^{\alpha_{V,2}}-1=83\cdot2^1-1=5v_2\\
		V_3=415&=v_12^{\alpha_{V,3}}-1=13\cdot2^5-1=5v_3
	\end{array}\hspace{1em}
	\begin{array}{ll}
		W_1=135&=w_22^{\alpha_{W,1}}-1=17\cdot2^3-1=5w_1\\
		W_2=85&=w_32^{\alpha_{W,2}}-1=43\cdot2^1-1=5w_2\\
		W_3=215&=w_12^{\alpha_{W,3}}-1=27\cdot2^3-1=5w_3
	\end{array}
	\]
	The result of these substitutions is:
	\[
	33\cdot\cancel{2^1}\cdot83\cdot\cancel{2^1}\cdot13\cdot\cancel{2^5}\cdot135\cdot85\cdot215=65\cdot165\cdot415\cdot17\cdot\cancel{2^3}\cdot43\cdot\cancel{2^1}\cdot27\cdot\cancel{2^3}
	\]
	Since the sum of exponents $\alpha_{V,i}$ and $\alpha_{W,i}$ are equal, we can cancel out all powers of two and obtain:
	\[
	v_2v_3v_1W_1W_2W_3=33\cdot83\cdot13\cdot135\cdot85\cdot215=65\cdot165\cdot415\cdot17\cdot43\cdot27=V_1V_2V_3w_2w_3w_1
	\]
	This product equality becomes true $v_2v_3v_1\cdot k^3\cdot w_1w_2w_3=k^3\cdot v_1v_2v_3\cdot w_2w_3w_1$ when we set $V_i=kv_i$ and $W_i=kw_i$ (for $i=1,2,3$) which inevitably leads to the two corresponding cycles for $k=5$ that are already presented by table~\ref{table:known_cycles}.
\end{example}

Let us define the difference $\Delta=(1+\nicefrac{1}{V_1})(1+\nicefrac{1}{V_2})\cdots(1+\nicefrac{1}{V_m})-(1+\nicefrac{1}{W_1})(1+\nicefrac{1}{W_2})\cdots(1+\nicefrac{1}{W_n})$.
We know that if this difference is zero, then we have found two cycles, as for example $0=(1+\nicefrac{1}{65})(1+\nicefrac{1}{165})(1+\nicefrac{1}{415})-(1+\nicefrac{1}{135})(1+\nicefrac{1}{85})(1+\nicefrac{1}{215})$.
Can you identify empirically some set pairs $\{V_1,V_2,\ldots,V_m\}$ and $\{W_1,W_2,\ldots,W_n\}$, where the difference is not necessarly zero but a whole number, e.g. $\Delta=1,2,3,\ldots$?
