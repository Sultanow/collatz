\chapter{Engel Expansions}
\label{ch:maximizing_target_node}

\section{Engel expansions maximize the node $v_{n+1}$}
A sequence $v_{n+1},v_n,\ldots,v_2,v_1$ describing a path in $H_{C,3}$ from $v_{n+1}$ down to $v_1$ allows at most one division by $2$ between two successive nodes. Dividing only once between two successive nodes, maximizes the $v_{n+1}$, but it is not obvious that this maximizes the product contained in condition~\ref{eq:condition_max}. Such a sequence forms the following ascending continued fraction (cf. also \cite[p.~11]{Ref_Laarhoven}):

\begin{equation}
\label{eq:asc_continued_fraction}
v_{n+1}=\cfrac{3\cfrac{3\cfrac{3\cfrac{3v_1+1}{2}+1}{2}+1}{2}+1}{2}\dotsb
=\frac{3^nv_1+\sum_{i=0}^{n-1}3^i2^{n-1-i}}{2^n}
=\frac{3^n(v_1+1)-2^n}{2^n}
\end{equation}

\par\medskip
The sum of the products of the powers of three and two, contained within the above term, can be simplified to the difference $3^n-2^n$ by converting the sum expression into the form $(x-1)(1+x+x^2+\cdots+x^{n-2}+x^{n-1})=x^n-1$ as follows:
\[
\frac{2^n}{2^n}(3-2)\sum_{i=0}^{n-1}3^i2^{n-1-i}
=\frac{2^n}{\cancel{2^{n-1}}}\cdot\frac{3-2}{2}\sum_{i=0}^{n-1}3^i2^{\cancel{n-1}-i}
=2^n\left(\frac{3}{2}-1\right)\sum_{i=0}^{n-1}\left(\frac{3}{2}\right)^i
=2^n\left(\left(\frac{3}{2}\right)^n-1\right)
\]

\begin{example}
	A concrete example for such a sequence is $v_1=31$, $v_2=47$, $v_3=71$, $v_4=107$, $v_5=161$. And, to follow that example, we can calculate the label of the vertex $v_5$ in a straightforward way:
	\[
	v_5=v_{n+1}=\frac{3^4(31+1)-2^4}{2^4}=161
	\]
	By choosing a vertex $v_1=2^{n+1}-1$, we are able to infinitely generate sequences which each form an ascending continued fraction. As per equation~\ref{eq:asc_continued_fraction} the last member in this sequence is the odd labeled vertex $v_{n+1}=3^n\cdot2-1$.
\end{example}

\begin{remark}
	Ascending variants of a continued fraction, such as used in equation~\ref{eq:asc_continued_fraction}, shall not be confused with continued fractions as treated for example in \cite{Ref_Moore}, \cite{Ref_Hensley}, \cite{Ref_Borwe_etal}. These ascending continued fractions correspond to the so-called "Engel Expansions" \cite{Ref_Kraaikamp_Wu}.
\end{remark}

\par\noindent
As illustrated below, we can formulate the ascending continued fractions in a generalized fashion, whereas the analogy to \ref{eq:asc_continued_fraction} is given by $b_1=b_2=b_3=b_4=2$ and $a_1=3^0$, $a_2=3^1$, $a_3=3^2$ and $a_4=3^3+3^4v_1$:
\[
\cfrac{a_1+\cfrac{a_2+\cfrac{a_3+\cfrac{a_4}{b_4}}{b_3}}{b_2}}{b_1}\dotsb=\frac{a_1}{b_1}+\frac{a_2}{b_1b_2}+\frac{a_3}{b_1b_2b_3}+\frac{a_4}{b_1b_2b_3b_4}+\cdots
\]

\par\medskip
The generalized form of equation~\ref{eq:asc_continued_fraction} may be used to compute any of the above-named ascending continued fraction that has $a_i=k^{i-1}$, $b_i=b$ for $i\in\mathbb{N}$ and $a_n=k^{n-1}+k^nv_1$:

\par\medskip
\begin{equation}
\label{eq:generalized_asc_continued_fraction}
v_{n+1}=\frac{k^n(kv_1-bv_1+1)-b^n}{b^n(k-b)}
\end{equation}

\section{Include more divisions by two into an Engel expansion}
\label{sec:include_divisions_engel_expansion}
In order to calculate the largest possible $v_{n+1}$, we have thus far considered Engel expansions which contain only $n$ division by two within a Collatz sequence of $n+1$ members. In the following, we include $m$ additional divisions by two and thus a total of $m+n$ divisions. Now we examine two corner cases:
\begin{itemize}
	\item the one where we do the additional $m$ divisions by $2$ at the end and
	\item the one where we do these additional divisions at the very beginning.
\end{itemize}

\par\noindent
\textbf{The first case} is our starting point to examine how swapping a division by two affects the node $v_{n+1}$. For this, let us compare the Engel expansion where we divide by $2^m$ afterwards with one where we divide by $2$ in the penultimate step and by $2^{m-1}$ in last step. One can immediately recognize the following inequality with a mere look:
\[
\cfrac{1+\cfrac{3+\cfrac{3^2+\cfrac{3^3+3^4v_1}{2}}{2}}{2}}{2\cdot2^m}
<
\cfrac{1+\cfrac{3+\cfrac{3^2+\cfrac{3^3+3^4v_1}{2}}{2}}{2\cdot\textcolor{red}{\mathbf{2}}}}{2\cdot2^{m-1}}
\]

To put it simply, in the expansion on the right side of the above-shown inequality we perform one division by two a little bit earlier as we do it in the expansion on the left side of the expansion. Almost all summands of both expansions cancel out each other:

\[
\frac{1}{2\cdot2^m}+\cancel{\frac{3}{2^2\cdot2^m}+\frac{3^2}{2^3\cdot2^m}+\frac{3^3+3^4v_1}{2^4\cdot2^m}}
<
\frac{1}{2\cdot2^{m-1}}+\cancel{\frac{3}{2^2\cdot\textcolor{red}{\mathbf{2}}\cdot2^{m-1}}+\frac{3^2}{2^3\cdot\textcolor{red}{\mathbf{2}}\cdot2^{m-1}}\frac{3^3+3^4v_1}{2^4\cdot\textcolor{red}{\mathbf{2}}\cdot2^{m-1}}}
\]

\par\bigskip\noindent
\textbf{The second case} deals with Engel expansions where we perform that additional $m$ divisions by two as early as possible. The resulting value decreases, when we make a division by two later:
\[
\cfrac{1+\cfrac{3+\cfrac{3^2+\cfrac{3^3+3^4v_1}{2\cdot2^{m-1}}}{2\cdot\textcolor{red}{\mathbf{2}}}}{2}}{2}
<
\cfrac{1+\cfrac{3+\cfrac{3^2+\cfrac{3^3+3^4v_1}{2\cdot2^m}}{2}}{2}}{2}
\]

Also here almost all summands of both Engel expansions cancel each other out:

\[
\cancel{\frac{1}{2}+\frac{3}{2^2}}+\frac{3^2}{2^3\cdot\textcolor{red}{\mathbf{2}}}+\cancel{\frac{3^3+3^4v_1}{2^4\cdot\textcolor{red}{\mathbf{2}}\cdot2^{m-1}}}
<
\cancel{\frac{1}{2}+\frac{3}{2^2}}+\frac{3^2}{2^3}+\cancel{\frac{3^3+3^4v_1}{2^4\cdot2^m}}
\]

\par\medskip
While the first case minimizes the value of the node $v_{n+1}$, the second case maximizes it. The difference between the maximum and the minimum is given by the following equation:

\begin{flalign*}
	&\frac{3^{n-1}\left(\frac{3v_1+1}{2\cdot2^m}+1\right)-2^{n-1}}{2^{n-1}}-\frac{3^n\left(v_1+1\right)-2^n}{2^{n+m}}\\
	=&\frac{3^{n-1}\cdot\left(3v_1+1+2^{m+1}\right)-2^{n-1}\cdot2^{m+1}-3^n\left(v_1+1\right)+2^n}{2^{m+1}\cdot2^{n-1}}\\
	=&\frac{3^{n-1}+3^{n-1}\cdot2^{m+1}-2^{n+m}-3^n+2^n}{2^{n+m}}=\frac{3^{n-1}-3\cdot3^{n-1}+3^{n-1}\cdot2^{m+1}-2^{n+m}+2^n}{2^{n+m}}\\
	=&\frac{-2\cdot3^{n-1}+3^{n-1}\cdot2^{m+1}-2^{n+m}+2^n}{2^{n+m}}=\frac{\left(2\cdot3^{n-1}-2^n\right)\left(2^m-1\right)}{2^n\cdot2^m}\\
	=&\left(\frac{3^{n-1}}{2^{n-1}}-1\right)\left(1-\frac{1}{2^m}\right)
\end{flalign*}

This has the consequence that for a given sequence consisting of $n+1$ members, between which a total of $n+m$ divisions have taken place, the permutation of these divisions has a very limited effect on the node $v_{n+1}$ as described by theorem~\ref{theo:permutation}.

\par\medskip
\begin{theorem}
	\label{theo:permutation}
	Let $v_{n+1},v_n,\ldots,v_2,v_1$ be a sequence in which a total of $n+m$ divisions took place (a path in which a total of $n+m$ edges has been contracted). No matter how these divisions are permuted, i.e. performed sooner or later, the node $v_{n+1}$ can differ at most by the following product:
	\[
	\left(\frac{3^{n-1}}{2^{n-1}}-1\right)\left(1-\frac{1}{2^m}\right)
	\]
\end{theorem}

\section{The product in the condition for alpha's upper limit}
Let us take a closer look at the product contained in condition~\ref{eq:condition_max} for the case $k=3$ and use the ascending continued fractions for examining this product. The exciting question is, does this product have a limit value even in the case where we only contract a single edge between successive nodes? Setting the according sequence, which maximizes $v_{n+1}$, into the product expressed by condition~\ref{eq:condition_max}, we obtain a product that is limited, or to be more specific, which in the worst case $v_1=1$ converges (for $n$ to infinity) towards $2$:
\begin{equation}
\label{eq:product_simplification_k3}
\prod_{i=1}^{n}\left(1+\frac{1}{3v_{i}}\right)
=\prod_{i=1}^{n}\left(1+\frac{1}{3\frac{3^{i-1}(v_1+1)-2^{i-1}}{2^{i-1}}}\right)
=\prod_{i=1}^{n}\frac{3^i(v_1+1)-2^i}{3^i(v_1+1)-3*2^{i-1}}
=\frac{1}{v_1}-\frac{1}{v_1}\left(\frac{2}{3}\right)^n+1
\end{equation}

The above-illustrated last forming step, simplifies this product significantly into an expression waiving a product formulation. A detailed breakdown including all intermediate steps of this simplification is shown in the appendix~\ref{appx:product_simplification_k3}. The correctness of this simplification can be proven inductively too, which we detail in appendix~\ref{appx:proof_product_simplification_k3}. The most important and the most interesting aspect of this result is, that the above simplified term cannot exceed the value $2$, whatever you choose to insert into $n$ or into $v_1$:
\[
\frac{1}{v_1}-\frac{1}{v_1}\left(\frac{2}{3}\right)^{n+1}+1<2
\]

As demonstrated above, since the product cannot exceed the value $2$, the logarithmic product expression in the condition~\ref{eq:condition_max} cannot exceed the value one and this condition becomes a consistently true statement:
\[
n\log_23-\lfloor n\log_23\rfloor<2-1
\]

Thus, for $k=3$ the condition~\ref{eq:condition_max} for alphas's upper limit is met for all sequences that maximize $v_{n+1}$.

\section{Include additional divisions into the product}
How does the product, contained in condition~\ref{eq:condition_max} look if we include the additional $m$ divisions into the Engel expansion as per section~\ref{sec:include_divisions_engel_expansion}? To answer this question, we consider the sequence $v_{n+1},v_n,v_{n-1},\ldots,v_2,v_1$ and we set $v_2=\frac{3v_1+1}{2\cdot2^{m}}$. Then reusing the continued fraction given by equation~\ref{eq:asc_continued_fraction}, we obtain:

{\setlength{\jot}{1.2em}
\begin{flalign}
\label{eq:asc_continued_fraction_m}
v_{n+1}&=\cfrac{3\cfrac{3\cfrac{3\cfrac{3v_1+1}{2\cdot2^m}+1}{2}+1}{2}+1}{2}\dotsb
=\cfrac{3\cfrac{3\cfrac{3v_2+1}{2}+1}{2}+1}{2}\dotsb
=\frac{3^{n-1}(v_2+1)-2^{n-1}}{2^{n-1}}\\
\notag
&=\frac{3^{n-1}(\frac{3v_1+1}{2\cdot2^{m}}+1)-2^{n-1}}{2^{n-1}}=\frac{3^nv_1+3^{n-1}+3^{n-1}2^{m+1}}{2^{m+n}}-1
\end{flalign}}

\par\noindent
The product will be calculated by using equation~\ref{eq:product_simplification_k3}:
\begin{flalign}
\label{eq:product_k3_m}
\prod_{i=1}^{n}\left(1+\frac{1}{3v_{i}}\right)&=\left(1+\frac{1}{3v_1}\right)\cdot\prod_{i=2}^{n}\left(1+\frac{1}{3v_{i}}\right)\\
\notag
&=\left(1+\frac{1}{3v_1}\right)\cdot\prod_{i=1}^{n-1}\left(1+\frac{1}{3v_{i+1}}\right)=\left(1+\frac{1}{3v_1}\right)\cdot\left(\frac{1}{v_2}-\frac{1}{v_2}\left(\frac{2}{3}\right)^{n-1}+1\right)
\end{flalign}

\par\noindent
Finally substituting $v_2=\frac{3v_1+1}{2\cdot2^{m}}$ into equation~\ref{eq:product_k3_m} leads to the simplified formula of the product:
\begin{equation}
\label{eq:product_simplification_k3_m}
\prod_{i=1}^{n}\left(1+\frac{1}{3v_{i}}\right)=\left(1+\frac{1}{3v_1}\right)\cdot\frac{1-\left(\frac{2}{3}\right)^{n-1}+v_2}{v_2}=\frac{1+2^{m+1}}{3v_1}-\frac{2^m}{v_1}\left(\frac{2}{3}\right)^n+1
\end{equation}

\begin{example}
	An example provides the sequence $v_1=661$, $v_2=31$, $v_3=47$, $v_4=71$, $v_5=107$. When we input $v_1=661$ with $m=5$ and $n=4$ into equation~\ref{eq:asc_continued_fraction_m} we retrieve the label of the vertex $v_5$:
	\[
	v_5=v_{n+1}=\frac{3^4\cdot661+3^3+3^3\cdot2^6}{2^9}-1=107
	\]
	In this sequence five $(m=5)$ additional divisions by two took place in the first step using $v_1$:
	\[
	\frac{3\cdot661-1}{2\cdot2^5}=v_2=31
	\]
	Let us now verify the formula for the product by taking this particular example. To this end, we input $v_1=661$ together with $m=5$ and $n=4$ into equation~\ref{eq:product_simplification_k3_m}:
	\[
	\left(1+\frac{1}{3\cdot661}\right)\left(1+\frac{1}{3\cdot31}\right)\left(1+\frac{1}{3\cdot47}\right)\left(1+\frac{1}{3\cdot71}\right)=\frac{1+2^{6}}{3\cdot661}-\frac{2^5}{661}\left(\frac{2}{3}\right)^4+1=1.023215853
	\]
\end{example}

\section{Condition for a limited growth of the Engel expansion}
\label{sec:condition_limited_growth}
Let us look now into the question of what condition must be met to prevent a greater growth than a decline in Collatz sequences. Specifically, we consider an Engel expansion comprising $n+1$ sequence members that include $m$ additional divisions by two at the beginning. The last member $v_{n+1}$ in such a sequence can be calculated by formula~\ref{eq:asc_continued_fraction_m}. In order to restrict the growth of this sequence, we require that the last member has to be smaller than the first one. For this we define the condition $v_{n+1}<v_1$:

\[
\frac{3^nv_1+3^{n-1}+3^{n-1}2^{m+1}}{2^{m+n}}-1<v_1
\]

By transforming this inequality, which is thoroughly described in the appendix~\ref{appx:condition_limited_growth} step by step, we obtain the condition:

\begin{equation}
\label{eq:condition_limited_growth}
	\frac{3^{n-1}\left(2^{m+1}-2\right)}{2^{m+n}-3^n}-1<v_1
\end{equation}

\section{Engel expansions maximize the product}
The question which sequence maximizes the target node $v_{n+1}$ ties into the question which sequence maximizes the product in the condition for cycle-alpha's upper limit given by equation~\ref{eq:condition_max}. The product formula that do not depend from all vertices $v_1,v_2,\ldots v_n$ has been evolved in appendix~\ref{appx:product_formula_depending_v1}. This formula depends only from $2^\alpha$, from the starting node $v_1$ and the target node $v_{n+1}$:

\[
\prod_{i=1}^{n}\left(1+\frac{1}{kv_i}\right)=\frac{2^{\alpha_1+\ldots+\alpha_n}v_{n+1}}{k^nv_1}
\]

In order to maximize this product, one needs to maximize the target node $v_{n+1}$, which is exactly what the Engel expansion does. Hence, for a given $v_1$, the Engel expansion is the worst case sequence maximizing the product in the condition for cycle-alpha's upper limit.
