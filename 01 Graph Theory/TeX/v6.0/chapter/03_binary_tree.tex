\chapter{Binary Collatz Tree}
\label{ch:binary_tree}

\section{Some essentials on binary trees}
A binary tree is a rooted tree, where each node has at most two immediate successors. Those nodes, from which no edge goes out downward, are called leaves, the others are called internal nodes. In a full binary tree, all internal nodes have exactly two children \cite[p.~102]{Ref_Higham_2015}. Full binary trees have an odd number $2n+1$ of nodes. Of these $n+1$ are leaves and $n$ are inner nodes \cite[p.~134]{Ref_Kersting_Wakolbinger_2008}. Each node in a binary tree has a left subtree and a right subtree, which is why a binary tree is inherently recursive, since the left and right subtrees of the root are themselves binary trees \cite[p.~246-247]{Ref_Mazur_2010}. As it often pops up in combinatorial problems, the famous $n$-th Catalan number, named after the Belgian mathematician Eugène Catalan, comes in connection with binary trees into play. For $n\ge1$ it specifies the number of binary trees on $n$ vertices \cite[p.~247]{Ref_Mazur_2010}:
\[
B_n=\sum_{i=0}^{n-1}B_iB_{n-1-i}=\sum_{i=1}^{n}B_{i-1}B_{n-i}=\frac{1}{n+1}\binom{2n}{n}
\]

There is an interesting property that trees exhibit regarding abstract algebra. Let's have a look at the algebraic structure of magmas. Consider an element $x$ of a magma $(M,*)$ which is an iterated product of other elements in $M$. Such an element can be described by a planar (no edges cross each other) rooted binary tree whose $n$ leaves are labelled by these other elements $x_1,\ldots,x_n\in M$ \cite[p.~96]{Ref_Kalka_2016}.

Binary trees make well-suited data structures for storing information. With about $2^m$ data points (nodes), a search of a binary tree takes only about $m$ steps, compared to about $2^{m-1}$ steps which are required to search a simple list \cite[p.~84]{Ref_Benjamin_2009}.

\section{Transforming the Collatz tree into a binary tree}
Jan Kleinnijenhuis and Alissa M. Kleinnijenhuis \cite{Ref_Kleinnijenhuis_2020a} introduced a binary tree $T_{\ge0}$ by transforming the original Collatz tree $H_U$ into the Syracuse tree $H_{C,3}$, which in turn is transformed into the binary tree $T_{\ge0}$ as described next. The edges are changed according to the following procedure: whenever a parent node $w$ has edges to its child nodes $v_0,v_1,\ldots,v_n$, on the tree $H_{C,3}$, we draw an edge from $w$ to $v_0$, and edges from $v_i$ to $v_{i+1}$ for each $i=1,\ldots,n-1$, in the binary new tree. Note that the nodes $v_1,v_2,\ldots,v_n$ are sorted in increasing order of label $v_0<v_1<\ldots<v_n$, which is already given by \ref{eq:n_fold_right_sibling_k}. Figure~\ref{fig:bt3} and \ref{fig:bt3_rot} display that tree -- once in our standard layout and once reversed (from bottom to top).

\begin{figure}[H]
	\includegraphics[width=1.00\textwidth]{figures/bt_3_t0.png}
	\caption{The Collatz Tree transformed to the binary tree $T_{\ge0}$}
	\label{fig:bt3}
\end{figure}

\vspace{-2em}
\begin{figure}[H]
	\includegraphics[width=1.00\textwidth]{figures/bt_3_t0_rot.png}
	\caption{The binary tree $T_{\ge0}$ with \textit{bottom-to-top} layout orientation}
	\label{fig:bt3_rot}
\end{figure}

\begin{remark}
	To clarify the terminology, it should be mentioned that Jan and Alissa M. Kleinnijenhuis in their manuscripts \cite{Ref_Kleinnijenhuis_2020a}, \cite{Ref_Kleinnijenhuis_2020b} denote the original Collatz tree $T_C$ while we call it $H_U$. They denote the Syracuse Tree $T_T$ which in our nomenclature is referred to as $H_{C,3}$.
\end{remark}

Nodes that are highlighted orange in figures~\ref{fig:bt3},~\ref{fig:bt3_rot} are called \textit{prunable} and they are exactly those nodes resulting as output of the \textit{Rightward} function. For navigating within this binary tree, Jan Kleinnijenhuis and Alissa M. Kleinnijenhuis \cite{Ref_Kleinnijenhuis_2020a} defined an \textit{Upward} function $U(n)$ and a \textit{Rightward} function $R(n)$ as follows:

\begin{equation}
\label{eq:bintree_3_rightward_upward}
\setlength{\arraycolsep}{1.6em}
\begin{array}{cc}
U(n)=\begin{cases}
        4n+1	&	n\equiv 1\pmod 6\\
        16n+5	&	n\equiv 5\pmod 6
    \end{cases} &
R(n)=\begin{cases}
    \nicefrac{(2^2n-1)}{3}	&	n\in[1]_{18}\cup[13]_{18}\\
    \nicefrac{(2^3n-1)}{3}	&	n\in[5]_{18}\\
    \nicefrac{(2^4n-1)}{3}	&	n\in[7]_{18}\\
    \nicefrac{(2^1n-1)}{3}	&	n\in[11]_{18}\cup[17]_{18}
\end{cases}
\end{array}
\end{equation}

The domain and codomain of both functions consist of the two residue classes $[1]_6,[5]_6$, which form the multiplicative (cyclic) group $\mathbb{Z}^\ast_6=\{1,5\}=\left<5\right>$. Consequently, the domain and codomain exclude all integers divisible by $2$ and $3$, which is due to the fact that this binary tree (just like our tree $H_{C,3}$) does not contain even numbers and additionally all leaves -- namely those nodes labeled with an integer divisible by three -- were deleted. The function $U(n)$ is very similar to the function~\ref{eq:next_sibling_k3} and to the more general function~\ref{eq:n_fold_right_sibling_k} (when setting $n=1,k=3$) which both calculate the right-sibling of a given vertex. This is clear, since siblings (parallel) in $H_{C,3}$ are successors (serial) in the binary tree $T_{\ge0}$. In the end, for a node $v_0$ having a leaf as right-sibling in $H_{C,3}$, the function $U(v_0)$ is defined as $v_1=4v_0+1$ executed twice $v_1=4(4v_0+1)+1=16v_0+5$, because we must skip this leaf. Recall that all leafs in $H_{C,3}$ are excluded from the binary tree without exception. For any $n\in[5]_6$ it applies that $U(n)\equiv16n+5\equiv\boldsymbol{1}\bmod(6)$ since $6\mid16n+5-\boldsymbol{1}$ resulting in $6\mid16(5+k\cdot6)+5-\boldsymbol{1}$, see \ref{eq:congruence}, and analogously for any $n\in[1]_6$ it applies that $U(n)\equiv4n+1\equiv\boldsymbol{5}\bmod(6)$ since $6\mid4n+1-\boldsymbol{5}$ resulting in $6\mid4(1+k\cdot6)+1-\boldsymbol{5}$. Therefore executing the Upward function twice in a row leads unconditionally to $U^2(n)=16(4n+1)+5=4(16n+5)+1=64n+21$.

\begin{remark}
	While we displayed trees from top to down, it is sometimes usual to draw trees in a bottom-to-top fashion as Kleinnijenhuis \cite{Ref_Kleinnijenhuis_2020b} do. The Rightward function corresponds to what we call left-child and the Upward function relates to the right-child which is commonly used in the context of binary trees \cite[p. 246]{Ref_Mazur_2010}.
\end{remark}

Jan and Alissa M. Kleinnijenhuis \cite{Ref_Kleinnijenhuis_2020a} defined the set $N(T_C)=N(H_U)$ that contains the labels of all nodes, to which a path from the root in $H_U$ exists, in other words, this set contains all integers $n$ for which the orbit of $n$ under the (uncompressed) Collatz function~\ref{eq:func_collatz} converges to $1$. Furthermore they introduced $S_{\ge0}$ as the node set containing integers that are neither divisible by $2$ nor by $3$. The set $S_{-1}$ comprises on the contrary all numbers, which are divisible by $2$ or $3$. In order to comprehend the structure of these sets $S$, let us take a look at the following list showing which tree includes which node set, see also the ancillary files of \cite{Ref_Kleinnijenhuis_2020a}, \cite{Ref_Kleinnijenhuis_2020b}:

\[\arraycolsep=0.6em\def\arraystretch{1.4}
\begin{array}{llll}
\text{Original Collatz tree} & N(T_C)=N(H_U)&=&\mathbb{N^+} \hspace{0.6em}\text{if the Collatz conjecture holds}\\
\text{Syracuse tree} & N(T_T)=N(H_{C,3})&=&N(T_C)\setminus2\mathbb{N}\\
\text{Binary tree}\hspace{0.6em}T_{\ge0}& N(T_{\ge0})=S_{\ge0}&=&N(T_C)\setminus S_{-1}\hspace{2.1em}=S_{0}\cup S_{1}\cup S_{2}\ldots\\
\text{Binary tree}\hspace{0.6em}T_{\ge1}& N(T_{\ge1})=S_{\ge1}&=&N(T_C)\setminus \bigcup_{i=-1}^{0}S_i=S_{1}\cup S_{2}\cup S_{3}\ldots\\
\text{Binary tree}\hspace{0.6em}T_{\ge j} & N(T_{\ge j})=S_{\ge j}&=&N(T_C)\setminus \bigcup_{i=-1}^{j-1}S_i=\bigcup_{i=j}^{\infty}S_i
\end{array}
\]

\par\medskip
Let us describe these sets using multiplicative groups. The set $S_{\ge0}=\mathbb{Z}^\ast_6$ can be understood as the multiplicative group modulo $6$ and the set $S_{-1}=\mathbb{Z}/6\mathbb{Z}\setminus\mathbb{Z}^\ast_6=\{0,2,3,4\}$ as the set of all non-invertible elements (non-units) of $\mathbb{Z}/6\mathbb{Z}$.

The set $S_0$ consists of all nodes resulting as output of $R(n)$ within the binary tree $T_{\ge0}$. These are the orange highlighted nodes displayed by figures~\ref{fig:bt3},~\ref{fig:bt3_rot}. In other words, $S_0$ is the codomain of the function $R(n)$ operating on nodes within $T_{\ge0}$. The binary tree $T_{\ge0}$ can be transformed to a (pruned) binary tree $T_{\ge1}$. For this, the prunable nodes will be deleted and their neighbors reconnected. The upward neighbor of a pruned node will then be identified as pruning candidate for a later transformation of the resulting tree $T_{\ge1}$ to a more pruned tree $T_{\ge2}$.

The set $S_1$ contains all nodes that are (as per the description above) identified as pruning candidates for the next transformation of $T_{\ge1}$ to $T_{\ge2}$. After having transformed $T_{\ge1}$ to $T_{\ge2}$, the more pruned binary tree $T_{\ge2}$ contains nodes that are identified as pruning candidates for another upcoming transformation of $T_{\ge2}$ to $T_{\ge3}$ -- these nodes are elements of the set $S_2$. This pruning algorithm is repeatedly applied in the same pattern. And in this way we obtain the sets $S_1,S_2,S_3,\ldots$ and so forth. Generally, we can write these sets in the form $S_j=\{n\in N(T_{j-1})\mid U^{-j}(n)\in S_0\}$. Kleinnijenhuis found out that the codomain $\mathbb{N}^U$ of the Upward function contains $5$ residue classes modulo $96$, namely $\{5, 29, 53, 77, 85\}=\mathbb{N}^U$ and the codomain $\mathbb{N}^R$ of the Rightward function comprises $27$ residue classes modulo $96$, namely $\{1, 7, 11, 13, 17, 19, 23, 25, 31, 35, 37, 41, 43, 47, 49, 55, 59, 61, 65, 67, 71, 73, 79, 83, 89, 91, 95\}=\mathbb{N}^R$. The union of both sets $\mathbb{N}^U\cup\mathbb{N}^R$ forms the non-cyclic multiplicative group $\mathbb{Z}^\ast_{96}$, whose generating set is $\{5, 17, 31\}$ (see \cite{Ref_Lang_2017}, \cite{Ref_OESIS_A033949}). All elements of the Upward function's codomain have the same remainder $5$ when divided by $8$.

For each subset $X$ of a group $G$, the intersection over all subgroups (of $G$) that contain this subset $X$ is \cite[p.~34]{Ref_Karpfinger_Meyberg_2017}:
\[
\left<X\right>=\bigcap_{X\subseteq U\le G}U
\]

Firstly it applies $\left<X\right>\le G$ meaning that this intersection is again a subgroup of $G$. It is generated by the \textit{generating set} $X$ and it is the smallest subgroup of $G$ containing every element of $X$ \cite[p.~35]{Ref_Karpfinger_Meyberg_2017}. Secondly, $\left<X\right>\subseteq U$ for each subgroup $U$ (of $G$) containing $X$.  Thirdly, when there is only a single element $x$ in $X$, then $\left<X\right>$ is usually written as $\left<x\right>$ and in this case, $\left<x\right>$ is the cyclic subgroup of $G$ -- such situations we have already seen in section~\ref{sec:left_child_right_sibling_3}. Let us refer back to $\mathbb{Z}^\ast_{96}$. In this example, $\left<\{5,17,31\}\right>$ is the subgroup generated by $\{5,17,31\}$ and therefore every element of $\mathbb{Z}^\ast_{96}$ is of the form $5^l17^m31^n$ where $l\in\{0,1,\ldots,7\}$ because the element $5$ has order $8$, and similarly $m,n\in\{0,1\}$ since both elements $17$ and $31$ have order $2$. Non-cyclic groups can be cyclic decomposed, which is detailed by Gallian and Rusin \cite{Ref_Gallian_Rusin_1980} and Cheng \cite{Ref_Cheng_1989} using the concept of the external direct product \cite[p.~79]{Ref_Karpfinger_Meyberg_2017}, \cite[p.~156]{Ref_Gallian} and the internal direct product \cite[p.~80]{Ref_Karpfinger_Meyberg_2017}, \cite[p.~183]{Ref_Gallian}. A comprehensive table of cyclic decompositions of multiplicative non-cyclic groups of integers modulo $n$ up to $n=130$ is provided by Wolfdieter Lang \cite{Ref_Lang_2017}.

% http://mathonline.wikidot.com/the-internal-direct-product-of-two-groups

Let us take a closer look at the (cyclic) multiplicative group $\mathbb{Z}^\ast_{18}=\{1,5,7,11,13,17\}=\left<5\right>$ which has an order $ord(\mathbb{Z}^\ast_{18})=6$. Having the generator $5$ coprime to the modulus $18$, we obtain the congruence $5^{\phi(18)}\equiv1\pmod{18}$ in accordance with Euler's theorem \ref{eq:eulers_theorem}. This allows us to infer from $5^6\equiv5^{6(n+1)}\equiv5^j5^{6n+6-j}\equiv1\pmod{18}$ the congruences given by \ref{eq:homomorphism_congruences} (on the left).

If a natural number divides another, $m\mid n$, as in our case $3\mid18$, then for two integers $a,b$ the following implication holds, see \cite[p.~21]{Ref_Mueller-Stach_2011}:

\begin{equation}
	\label{eq:reduce_modulus}
	a\equiv b\pmod n\rightarrow a\equiv b\pmod m
\end{equation}

This means in our case $w\cdot5^{6n+6}\equiv 1\pmod{18}\rightarrow w\cdot5^{6n+6}\equiv 1\pmod 3$.
In fact, Euler's theorem (\ref{eq:eulers_theorem}) gives us two congruences $5^{\phi(18)}\equiv1\pmod{18}$ and $2^{\phi(3)}\equiv1\pmod{3}$. The latter is obvious, because $\mathbb{Z}_3^\ast=\left<2\right>$. Since $\phi(18)=6$ and $\phi(3)=2$, every power of five with an exponent divisible by $6$ and every power of two with an exponent divisible by 2 belong to the residue classes $[1]_{18}$ and $[1]_3$.

Because of $[1]_{18}=[5^0]_{18},\ldots,[13]_{18}=[5^4]_{18}$ and finally $[11]_{18}=[5^5]_{18}$ we obtain the congruences on the left side in \ref{eq:homomorphism_congruences}. The exponents are indicated by the $j$. For example, it follows from $w\in[13]_{18}$ that $[w\cdot5^{6n+2}]_{18}=[w]_{18}\cdot[5^{6n+2}]_{18}=[13]_{18}\cdot[5^{6n+2}]_{18}=[5^4]_{18}\cdot[5^{6n+2}]_{18}=[5^{6n+6}]_{18}=[1]_{18}$, because $6n+6$ is divisible by $6$ (Euler's theorem).

From the equality of the residue classes modulo $18$ and modulo $3$ (according to equation~\ref{eq:reduce_modulus}) it follows for $w\in[13]_{18}$ from $[w]_{18}\cdot[5^{6n+2}]_{18}=[13]_{18} \cdot[5^{6n+2}]_{18}=[1]_{18}$ the following equation:

\begin{equation}
\label{eq:mod_18_3}
[1]_3=[w]_3\cdot[5^{6n+2}]_3=[13]_3\cdot[5^{6n+2}]_3
\end{equation}

Just replacing $18$ with $3$, that is the homomorphism given by the map $f:\mathbb{Z}^\ast_n\rightarrow \mathbb{Z}^\ast_m$ with $f(r\bmod n)=r\bmod m$ as long as $\gcd(r\bmod n,n)=1$ leads to $\gcd(r\bmod m,m)=1$. By Euclid, we know that in the case $r$ is coprime to $n$ then it is also coprime to every factor $m$ of $n$. That is why a homomorphism exist to the congruences shown in \ref{eq:homomorphism_congruences} (on the right).

Using the fact that $[5^{6n}]_{18}=[5^0]_{18}=[1]_{18}$ and thus $[5^{6n+k}]_{18}=[5^k]_{18}=[5]_{18}^k=[5]_3^k$, we obtain from equation~\ref{eq:mod_18_3}:

\[
[1]_3=[13]_3\cdot[5]_3^2=[13]_3\cdot[2]_3^2
\]

Therefore $[13]_{18}\cdot4=[13]_3\cdot4\equiv1\pmod{3}$. The third and the last row in \ref{eq:homomorphism_congruences} are obtained in the same way. The powers of two that appear in the other rows result from the fact that we can always add or remove even powers of two because $[2^2]_3=[1]_3$.

Based on equation~\ref{eq:congruence_reduction} we can state that $a\equiv b\pmod m$ implies $(a+m)\equiv b\pmod m$. Let us set $a=w\cdot2^{2n+2}$ and $b=1$ and $m=3$, then we obtain $(w\cdot2^{2n+2}+3)\equiv1\pmod3$ and using a factor $i\in\mathbb{N}$ we obtain the more general congruence $(w\cdot2^{2n+2}+3i)\equiv1\pmod3$. As a consequence the congruences in \ref{eq:homomorphism_congruences} are true, while $w\cdot5^{6n+6}=w\cdot2^{2n+2}+3\cdot i$ or rather while $3\mid(5^{6n+6}-2^{2n+2})$. These conditions continue $3\mid(5^{6n+5}-2^{2n+3})$ and $3\mid(5^{6n+4}-2^{2n+4})$ and so forth.

\begin{equation}
\label{eq:homomorphism_congruences}
\begin{array}{llllll}
	j=0, & w\in [1]_{18} & \hspace{1em} w\cdot5^{6n+6}&\equiv1\pmod{18}&\hspace{4em}w\cdot2^{2n+2} &\equiv1\pmod{3}\\
	j=1, & w\in [5]_{18} & \hspace{1em} w\cdot5^{6n+5}&\equiv1\pmod{18}&\hspace{4em}w\cdot2^{2n+3} &\equiv1\pmod{3}\\
	j=2, & w\in [7]_{18} & \hspace{1em} w\cdot5^{6n+4}&\equiv1\pmod{18}&\hspace{4em}w\cdot2^{2n+4} &\equiv1\pmod{3}\\
	j=3, & w\in [17]_{18} & \hspace{1em} w\cdot5^{6n+3}&\equiv1\pmod{18}&\hspace{4em}w\cdot2^{2n+1} &\equiv1\pmod{3}\\
	j=4, & w\in [13]_{18} & \hspace{1em} w\cdot5^{6n+2}&\equiv1\pmod{18}&\hspace{4em}w\cdot2^{2n+2} &\equiv1\pmod{3}\\
	j=5, & w\in [11]_{18} & \hspace{1em} w\cdot5^{6n+1}&\equiv1\pmod{18}&\hspace{4em}w\cdot2^{2n+1} &\equiv1\pmod{3}
\end{array}
\end{equation}

The inverse upward function, which can be understood as a downward function $U^{-1}(n)=D(n)$ is defined as follows:

\[
D(n)=U^{-1}(n)=\begin{cases}
	\nicefrac{n-1}{4}	&	n\equiv 5\pmod{24}\\
	\nicefrac{n-5}{16}	&	n\equiv 85\pmod{96}
\end{cases}
\]

The modular conditions are deduced as follows: We have $D(n)=\nicefrac{n-1}{4}$ if $\nicefrac{n-1}{4}\equiv1\pmod6$ and $D(n)=\nicefrac{n-5}{16}$ if $\nicefrac{n-5}{16}\equiv5\pmod6$. For $a,b\in\mathbb{Z}$ and $n,m\in\mathbb{N}$ we can apply the following modular arithmetic rule \cite[p.~21]{Ref_Mueller-Stach_2011}:

\begin{equation}
a\equiv b\pmod n \leftrightarrow m\cdot a\equiv m\cdot b\pmod{m\cdot n}
\end{equation}

This leads to $D(n)=n-1$ if $n-1\equiv4\pmod{24}$ and $D(n)=n-5$ if $n-5\equiv80\pmod{96}$.

Now we make the use of another modular arithmetic rule, which for two given congruences $a\equiv b\pmod n$ and $c\equiv d\pmod n$ states that \cite[p.~19]{Ref_Mueller-Stach_2011}:

\begin{equation}
	a+c\equiv b+d\pmod n\hspace{2em} \text{and}\hspace{2em}a\cdot c\equiv b\cdot d\pmod n
\end{equation}

This finally leads to $D(n)=n-1$ if $n\equiv5\pmod{24}$ and $D(n)=n-5$ if $n\equiv85\pmod{96}$.

The rightward function that operates on the tree of pruning level $1$, thus the tree $T_{\ge1}$, is can be expressed as the right-to-left composition $R_{\ge1}(n)=U\circ R\circ U^{-1}(n)=U\circ R\circ D(n)$ and it is defined as follows:

\[
R_{\ge1}(n)=\begin{cases}
	\nicefrac{(8D(n)-1)}{3}	&	2D(n)\equiv4\bmod{18}\wedge D(n)\in[11]_{18}\cup[17]_{18}\\
	\nicefrac{(16D(n)-1)}{3}	&	4D(n)\equiv4\bmod{18}\wedge D(n)\in[11]_{18}\cup[13]_{18}\\
	\nicefrac{(32D(n)-1)}{3}	&	2D(n)\equiv16\bmod{18}\wedge D(n)\in[11]_{18}\cup[17]_{18}\vee D(n)\in[5]_{18}\\
	\nicefrac{(64D(n)-1)}{3}	&	4D(n)\equiv16\bmod{18}\wedge D(n)\in[1]_{18}\cup[13]_{18}\vee D(n)\in[7]_{18}\end{cases}
\]

\newpage
Figure~\ref{fig:tree_transformations} shows the complete chain of tree transformations, beginning from the original Collatz tree, over the Syracuse tree to the binary tree and pruned ones.

% trim=left bottom right top
\begin{figure}[H]
	\includegraphics[trim=1.1cm 10cm 2.6cm 0.2cm, 
	width=1.00\textwidth,page=1]{figures/tree_transformations.pdf}
	\caption{Transformation chain, beginning from the original Collatz tree up to pruned binary trees}
	\label{fig:tree_transformations}
\end{figure}