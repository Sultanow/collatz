\chapter{Conclusion and Outlook}

\section{Summary}
We defined an algebraic graph structure that expresses the Collatz sequences in the form of a tree. Next, the vertex reachability properties were unveiled by examining the relationship between successive nodes in $H_C$. Moreover, we dealt with graphs that represent other variants of Collatz sequences, for instance $5x+1$ or $181x+1$. The interesting part of both variants just mentioned is that for these sequences the existence of cycles is known. This compact definitory digression serves as the basis for further investigations of the tree $H_C$.

\section{Further Research}
In subsequent studies, the properties of vertices in $H_C$ might be elaborated upon more closely by taking into account a vertex's label as well as its properties. In addition, future steps may include a detailed analysis of theorem~\ref{theo:2}.

In the next version of our manuscript we will take a more thorough look at the product expressed by condition~\ref{eq:condition_max} including the effort to show that (and why) the product always stays below a certain limit for any $k>1$.
