\chapter{Examination of \boldmath$H_{C,3}$}

\section{Which sequence is a worst case?}
\label{sec:worstcase_k3}
%Let us take a closer look at the product contained in condition~\ref{eq:condition_max} for the case $k=3$. The exciting main question is, does this product have a limit value even in the worst case?
Trying to find a worst case sequence means to search for a sequence of odd numbers that rises as high as possible. One could try the ascending sequence of odd integers $v_i=2i-1$ (beginning at $v_1=1$), but will find that for this case the product contained in condition~\ref{eq:condition_max} will not converge against a limit value (see appendix~\ref{appx:pochhammer}).

\par\medskip
A worst case sequence $v_{n+1},v_n,\ldots,v_2,v_1$ describing a path in $H_{C,3}$ from $v_{n+1}$ down to $v_1$ allows at most one division by $2$ between two successive nodes. This sequence forms the following ascending continued fraction (cf. also \cite[p.~11]{Ref_Laarhoven}):

\begin{equation}
\label{eq:asc_continued_fraction}
	v_{n+1}=\cfrac{3\cfrac{3\cfrac{3v_1+1}{2}+1}{2}+1}{2}\dotsb
	=\frac{3^nv_1+\sum_{i=0}^{n-1}3^i2^{n-1-i}}{2^n}
	=\frac{3^n(v_1+1)-2^n}{2^n}
\end{equation}

\par\medskip
The sum of the products of the powers of three and two, contained within the above term, can be simplified to the difference $3^n-2^n$ by converting the sum expression into the form $(x-1)(1+x+x^2+\cdots+x^{n-2}+x^{n-1})=x^n-1$ as follows:
\[
\frac{2^n}{2^n}(3-2)\sum_{i=0}^{n-1}3^i2^{n-1-i}
=\frac{2^n}{\cancel{2^{n-1}}}\cdot\frac{3-2}{2}\sum_{i=0}^{n-1}3^i2^{\cancel{n-1}-i}
=2^n\left(\frac{3}{2}-1\right)\sum_{i=0}^{n-1}\left(\frac{3}{2}\right)^i
=2^n\left(\left(\frac{3}{2}\right)^n-1\right)
\]

\begin{example}
A concrete example for such a sequence is $v_1=31$, $v_2=47$, $v_3=71$, $v_4=107$, $v_5=161$. And, to follow that example, we can calculate the label of the vertex $v_5$ in a straightforward way:
\[
v_5=v_{n+1}=\frac{3^4(31+1)-2^4}{2^4}=161
\]
\end{example}

\begin{remark}
Ascending variants of a continued fraction, such as used in equation~\ref{eq:asc_continued_fraction}, shall not be confused with continued fractions as treated for example in \cite{Ref_Moore}, \cite{Ref_Hensley}, \cite{Ref_Borwe_etal}. These ascending continued fractions correspond to the so-called "Engel Expansions" \cite{Ref_Kraaikamp_Wu}.
\end{remark}

\par\noindent
As illustrated below, we can formulate the ascending continued fractions in a generalized fashion, whereas the analogy to \ref{eq:asc_continued_fraction} is given by $b_1=b_2=b_3=b_4=2$ and $a_1=3^0$, $a_2=3^1$, $a_3=3^2$ and $a_4=3^3+3^4v_1$:
\[
\cfrac{a_1+\cfrac{a_2+\cfrac{a_3+\cfrac{a_4}{b_4}}{b_3}}{b_2}}{b_1}\dotsb=\frac{a_1}{b_1}+\frac{a_2}{b_1b_2}+\frac{a_3}{b_1b_2b_3}+\frac{a_4}{b_1b_2b_3b_4}+\cdots
\]

The generalized form of equation~\ref{eq:asc_continued_fraction} may be used to compute any of the above-named ascending continued fraction that has $a_i=k^{i-1}$, $b_i=b$ for $i\in\mathbb{N}$ and $a_n=k^{n-1}+k^nv_1$:
\begin{equation}
\label{eq:generalized_asc_continued_fraction}
v_{n+1}=\frac{k^n(kv_1-bv_1+1)-b^n}{b^n(k-b)}
\end{equation}

\section{The product in the condition for alpha's upper limit}
Taking the Engel expansion as worst case sequence and setting it into the product expressed by condition~\ref{eq:condition_max}, we obtain a product that is limited, or to be more specific, which in the worst case $v_1=1$ converges (for $n$ to infinity) towards $2$:
\begin{equation}
\label{eq:product_simplification_k3}
\prod_{i=1}^{n}\left(1+\frac{1}{3v_{i}}\right)
=\prod_{i=1}^{n}\left(1+\frac{1}{3\frac{3^{i-1}(v_1+1)-2^{i-1}}{2^{i-1}}}\right)
=\prod_{i=1}^{n}\frac{3^i(v_1+1)-2^i}{3^i(v_1+1)-3*2^{i-1}}
=\frac{1}{v_1}-\frac{1}{v_1}\left(\frac{2}{3}\right)^n+1
\end{equation}

The above-illustrated last forming step, simplifies this product significantly into an expression waiving a product formulation. A detailed breakdown including all intermediate steps of this simplification is shown in the appendix~\ref{appx:product_simplification_k3}. The correctness of this simplification can be proven inductively too, which we detail in appendix~\ref{appx:proof_product_simplification_k3}. The most important and the most interesting aspect of this result is, that the above simplified term cannot exceed the value $2$, whatever you choose to insert into $n$ or into $v_1$:
\[
\frac{1}{v_1}-\frac{1}{v_1}\left(\frac{2}{3}\right)^{n+1}+1<2
\]

For this reason, the logarithmic product expression in the condition~\ref{eq:condition_max} cannot exceed the value one, strictly speaking the worst case for that condition is:
\[
n\log_23-\lfloor n\log_23\rfloor<2-1
\]

Thus, we have proved that for $k=3$ the condition~\ref{eq:condition_max} for alphas's upper limit is always met.