\chapter{Conclusion and Outlook}

\section{Summary}
We defined an algebraic graph structure that expresses the Collatz sequences in the form of a tree. Next, the vertex reachability properties were unveiled by examining the relationship between successive nodes in $H_C$. Moreover, we dealt with graphs that represent other variants of Collatz sequences, for instance $5x+1$ or $181x+1$. The interesting part of both variants just mentioned is that for these sequences the existence of cycles is known. With regard to a proof of the Collatz conjecture, theorem~\ref{theo:2} and theorem~\ref{theo:3} seem promising. They serve as the basis for further investigations of the problem.

\section{Further Research}
In subsequent studies, the properties of vertices in $H_C$ might be elaborated upon more closely by taking into account a vertex's label as well as its properties. In addition, future steps may include a detailed analysis of theorem~\ref{theo:2} and theorem~\ref{theo:3}.