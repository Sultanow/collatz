\appendix
\chapter{Appendix}

\section{Sum of reciprocal vertices}
\label{appx:sum_reciprocal_vertices}
One condition deduced from theorem~\ref{theo:1} is the product condition~\ref{eq:condition_max}, which specifies the validity of the cycle-alpha's upper limit. This condition requires the sum $\frac{1}{kv_1}+\frac{1}{kv_2}+\frac{1}{kv_3}+\ldots$ to be limited. In order to formulate this sum independently from the successive vertices $v_2,v_3,\ldots$, we substitute these as follows:
\begin{flalign*}
v_1&=v_1\\
v_2&=\frac{kv_1+1}{2^{\alpha_1}}\\
v_3&=\frac{k^2v_1+k+2^{\alpha_1}}{2^{\alpha_1+\alpha_2}}\\
v_4&=\frac{k^3v_1+k^2+k\cdot2^{\alpha_1}+2^{\alpha_1+\alpha_2}}{2^{\alpha_1+\alpha_2+\alpha_3}}\\
\vdots\\
v_{n+1}&=\frac{k^nv_1+\sum_{j=1}^{n}k^{j-1}2^{\alpha_1+\ldots+\alpha_n-\sum_{l>n-j}\alpha_l}}{2^{\alpha_1+\ldots+\alpha_n}}
\end{flalign*}

\par\medskip
The sum of the reciprocal vertices can be expressed as a term that depends from $v_1$ and from the number of contracted edges, id est the number of dvisions by two, between two successive vertices $\alpha_1,\alpha_2,\alpha_3,\ldots$: 
\begin{equation*}
\sum_{i=1}^{n+1}\frac{1}{kv_i}=\frac{1}{k}\left(\frac{1}{v_1}+\sum_{i=1}^{n}\frac{1}{v_{i+1}}\right)=\frac{1}{k}\left(\frac{1}{v_1}+\sum_{i=1}^{n}\frac{2^{\alpha_1+\ldots+\alpha_i}}{k^iv_1+\sum_{j=1}^{i}k^{j-1}2^{\alpha_1+\ldots+\alpha_n-\sum_{l>i-j}\alpha_l}}\right)
\end{equation*}

\section{Simplying the product for $k=3$}
\label{appx:product_simplification_k3}
Below we will show the simplification of the product in the condition for alpha's upper limit, which has been performed by equation~\ref{eq:product_simplification_k3}:
\[
\prod_{i=1}^{n}\frac{3^i(v_1+1)-2^i}{3^i(v_1+1)-3*2^{i-1}}
=\frac{1}{v_1}-\frac{1}{v_1}\left(\frac{2}{3}\right)^n+1
\]

In fact, this product is a telescoping product. We factor out $\frac{1}{3^n}$, then shift the index in the product of the denominator by one to start with $i=0$, and use the product's telescopic property to cancel equal factors in numerator and denominator:
\begin{flalign*}
	&\prod_{i=1}^{n}\frac{3^i(v_1+1)-2^i}{3^i(v_1+1)-3*2^{i-1}}
	=\frac{1}{3^n}\prod_{i=1}^{n}\frac{3^i(v_1+1)-2^i}{3^{i-1}(v_1+1)-2^{i-1}}
	=\frac{1}{3^n}\frac{\prod_{i=1}^{n}\left(3^i(v_1+1)-2^i\right)}{\prod_{i=1}^{n}\left(3^{i-1}(v_1+1)-2^{i-1}\right)}\\
	=&\frac{1}{3^n}\frac{\prod_{i=1}^{n}\left(3^i(v_1+1)-2^i\right)}{\prod_{i=0}^{n-1}\left(3^i(v_1+1)-2^i\right)}
	=\frac{1}{3^n}\frac{3^n(v_1+1)-2^n}{(v_1+1)-1}
	=\frac{3^nv_1+3^n-2^n}{3^nv_1}
	=\frac{1}{v_1}-\frac{1}{v_1}\left(\frac{2}{3}\right)^n+1
\end{flalign*}

\section{Proving the product simplification for $k=3$ inductively}
\label{appx:proof_product_simplification_k3}
Using induction, we prove the simplification below that has been made by equation~\ref{eq:product_simplification_k3}:
\[
\prod_{i=1}^{n}\frac{3^i(v_1+1)-2^i}{3^i(v_1+1)-3*2^{i-1}}
=\frac{1}{v_1}-\frac{1}{v_1}\left(\frac{2}{3}\right)^n+1
\]

The base case $n=1$ is readily comprehensible and obviously correct:
\[
\prod_{i=1}^{1}\frac{3^i(v_1+1)-2^i}{3^i(v_1+1)-3*2^{i-1}}
=\frac{3(v_1+1)-2}{3(v_1+1)-3}
=\frac{1}{3v_1}+1
=\frac{1}{v_1}-\frac{1}{v_1}\left(\frac{2}{3}\right)+1
\]

The induction step is explained below, and here we arrive at a true statement too:
\begin{flalign*}
	\prod_{i=1}^{n+1}\frac{3^i(v_1+1)-2^i}{3^i(v_1+1)-3*2^{i-1}}&=\frac{3^{n+1}(v_1+1)-2^{n+1}}{3^{n+1}(v_1+1)-3*2^n}\prod_{i=1}^{n}\frac{3^i(v_1+1)-2^i}{3^i(v_1+1)-3*2^{i-1}}\\
	&=\frac{3^{n+1}(v_1+1)-2^{n+1}}{3^{n+1}(v_1+1)-3*2^n}\left(\frac{1}{v_1}-\frac{1}{v_1}\left(\frac{2}{3}\right)^n+1\right)\\
	&=\frac{3^{n+1}(v_1+1)-2^{n+1}}{3^{n+1}(v_1+1)-3*2^n}\cdot\frac{3^n-2^n+3^nv_1}{3^nv_1}\\
	&=\frac{3^{n+1}(v_1+1)-2^{n+1}}{\cancel{3^{n+1}(v_1+1)-3*2^n}}\cdot\frac{\cancel{3*(3^n-2^n+3^nv_1)}}{3*3^nv_1}\\
	&=\frac{1}{v_1}-\frac{1}{v_1}\left(\frac{2}{3}\right)^{n+1}+1
\end{flalign*}

%\section{An alternative proof for alpha's upper limit for $H_{C,1}$}
%\label{appx:proof_k1}
%We demonstrate that condition~\ref{eq:condition_max} is true for $k=1$. What makes this case so special and therefore so manageable is that the equation in theorem~\ref{theo:2} constantly yields $2^1$, whatever value we use for $n$. By setting $k=1$, the condition becomes reduced to:
%\begin{equation}
%\label{eq:condition_k1}
%\prod_{i=1}^{n}\frac{v_i+1}{v_i}<2^2
%\end{equation}
%One can see instantly that the condition~\ref{eq:condition_k1} above is met for $n=v_1=1$. This trivial cycle only includes the sole vertex $v_1=1$. The fact which causes a worst case sequence $v_n,v_{n-1},\ldots,v_2,v_1$ describing a path from $v_n$ to $v_1$ is precisely that between two successive nodes a division by two was only made once:
%\begin{equation}
%\label{eq:worst_case_1}
%\arraycolsep=1.4pt
%\begin{array}{llll}
%v_n&=2^{n-1}&\cdot\ (v_1-1)+1&\\
%v_{n-1}&=2^{n-2}&\cdot\ (v_1-1)+1&\\
%\vdots\\
%v_2&=2^1&\cdot\ (v_1-1)+1&=2v_1-1\\
%v_1&=2^0&\cdot\ (v_1-1)+1&=v_1
%\end{array}
%\end{equation}
%One example for such a sequence is $v_4=17,v_3=9,v_2=5,v_1=3$. It shall be mentioned that the sequence $v_1,2\cdot v_1-1,4\cdot v_1-3,\ldots$ is an increasing one for any $v_1>1$, which means $v_1<v_2<\ldots<v_{n-1}<v_n$. Why might such a sequence be referred to as worst case? Ultimately, it is because one needs to show that the product stays below the upper limit $2^2=4$. The smaller the values (labels) of the vertices, the larger the product. If we allowed additional divisions by $2$, the sequence would increase more steeply, the vertices' values would be larger and the product would consequently be smaller.

%Setting the worst case sequence $v_n=2^{n-1}(v_1-1)+1$ into the product \ref{eq:condition_k1} leads to the following product:
%\begin{equation}
%\label{eq:condition_k1_v1}
%\prod_{i=1}^{n}\frac{2^{i-1}(v_1-1)+2}{2^{i-1}(v_1-1)+1}
%\end{equation}
%As previously mentioned, we have to consider the worst case scenario, which results in the maximum product. We provoke the worst case if a vertex's value is as small as possible, which we achieve with the sequence $1,3,5,9,17,\ldots$ that is composed from two partial sequences, namely the one-element sequence $v_1=1$ and the sequence defined by \ref{eq:worst_case_1} starting with $v_1=3$. As product we then receive the composed product given below which must remain below the limit 4:
%\[
%\prod_{i=1}^{1}\frac{v_i+1}{v_i}\prod_{i=1}^{n}\frac{2^{i-1}(v_1-1)+2}{2^%{i-1}(v_1-1)+1}=2\prod_{i=1}^{n}\frac{2^i+2}{2^i+1}<4
%\]
%The first sub-product refers to \ref{eq:condition_k1} and comprises only a single iteration. We insert the value $v_1=1$ yielding a final result of $2$. The second sub-product is sourced from \ref{eq:condition_k1_v1} and has been simplified by setting $v_1=3$. We further facilitate this second sub-product as shown below:
%\begin{equation*}
%	\prod_{i=1}^{n}\frac{2^i+2}{2^i+1}=2^n\prod_{i=1}^{n}\frac{2^{i-1}+1}{2^i+1}=2^n\frac{(2^0+1)\cancel{(2^1+1)}\cancel{(2^2+1)}\cdots\cancel{(2^{n-1}+1)}}{\cancel{(2^1+1)}\cancel{(2^2+1)}\cdots\cancel{(2^{n-1}+1)}(2^n+1)}=\frac{2^{n+1}}{2^n+1}
%\end{equation*}

%The upper limit of this second sub-product is $2$ and consequently the entire product composed by both sub-products therefore converges from below towards $4$, which leads to our condition~\ref{eq:condition_k1} being fulfilled even in the worst case:
%\[
%\prod_{i=1}^{\infty}\frac{2^i+2}{2^i+1}=\lim_{n\to\infty}\frac{2^{n+1}}{2^n+1}=2
%\]

\section{Further worst-case studies}
\label{sec:worstcase_k3}
Regarding worst case scenarios, a distinction must be made between two basic cases:
\begin{itemize}
	\item The vertex $v_{n+1}$ becomes a maximum. This kind of worst case we have dealt with in chapter~\ref{ch:maximizing_target_node} and we used for proving cycle-alpha's upper limit in $H_{C,1}$ with section~\ref{sec:alphas_upper_limit_k_1}.
	\item The product in condition~\ref{eq:condition_max} and consequently the sum of reciprocal vertices, formulated in \ref{appx:sum_reciprocal_vertices}, becomes a maximum.
\end{itemize}
Trying to find a worst case that maximizes the product in condition~\ref{eq:condition_max} means to search for a sequence of odd numbers that rises as high as possible. One could try the ascending sequence of odd integers $v_i=2i-1$ (beginning at $v_1=1$), but will find that for this case the product will not converge against a limit value. This sequence (beginning at $v_1=1$) allow us to transform the product contained in condition~\ref{eq:condition_max} into a limit analyzable function using the Pochhammer’s symbol (sometimes referred to as the \textit{rising factorial} or \textit{shifted factorial}), which is denoted by $(x)_n$ and defined as follows \cite{Ref_Zwillinger_Kokoska}, \cite[p.~679]{Ref_Brychkov} and \cite[p.~1005]{Ref_Trott}:
\[
(x)_n=x(x+1)(x+2)\cdots(x+n-1)=\prod_{i=0}^{n-1}(x+i)=\prod_{i=1}^{n}(x+i-1)=\frac{\Gamma(x+n)}{\Gamma(x)}
\]

Setting $v_i=2i-1$ into the product expressed by condition~\ref{eq:condition_max} and setting $x=\frac{k+1}{2k}$ into Pochhammer’s symbol $(x)_n$ interestingly makes it possible for us to perform the following transformation:
\begin{equation}
\label{eq:pochhammer}
\prod_{i=1}^{n}\left(1+\frac{1}{kv_i}\right)
=\frac{\prod_{i=1}^{n}(kv_i+1)}{\prod_{i=1}^{n}kv_i}
=\frac{\prod_{i=1}^{n}\left(k(2i-1)+1\right)}{k^n\prod_{i=1}^{n}(2i-1)}
=\frac{2^{2n}n!}{(2n)!}\cdot\frac{\Gamma\left(\frac{k+1+2kn}{2k}\right)}{\Gamma\left(\frac{k+1}{2k}\right)}
\end{equation}

\begin{example}
	One simple example that is easy to recalculate may be provided by choosing $k=3$ and $n=4$:
	\[
	\left(1+\frac{1}{3*1}\right)\left(1+\frac{1}{3*3}\right)\left(1+\frac{1}{3*5}\right)\left(1+\frac{1}{3*7}\right)=1,6555=\frac{2^8*4!}{8!}\cdot\frac{\Gamma(\frac{14}{3})}{\Gamma(\frac{4}{6})}
	\]
\end{example}

The product in the numerator in equation~\ref{eq:pochhammer} will be transformed into a form that allows us to use the Pochhammer’s symbol:
\[\prod_{i=1}^{n}\left((2i-1)k+1\right)=2^nk^n\prod_{i=1}^{n}\frac{(2i-1)k+1}{2k}=2^nk^n\prod_{i=1}^{n}\frac{k+1+2ki-2k}{2k}=2^nk^n\prod_{i=1}^{n}\left(\frac{k+1}{2k}+i-1\right)\]

This product can be written now as $2^nk^n(x)_n$, whwereby $x=\frac{k+1}{2k}$:
\[\prod_{i=1}^{n}\left((2i-1)k+1\right)=2^nk^n\frac{\Gamma\left(\frac{k+1+2kn}{2k}\right)}{\Gamma\left(\frac{k+1}{2k}\right)}\]

We recall the basic fact that the product of even integers is given by $\prod_{i=1}^{n}2i=2^n\cdot n!$ and the product of odd integers is $\prod_{i=1}^{n}\left(2i-1\right)=1\cdot3\cdot5\cdot7\ldots=\frac{(2n)!}{2^n\cdot n!}$. Thus we can transform the product in the denominator in equation~\ref{eq:pochhammer} as follows:
\[\prod_{i=1}^{n}kv_i=k^n\prod_{i=1}^{n}v_i=k^n\prod_{i=1}^{n}(2i-1)=k^n\frac{(2n)!}{2^nn!}\]

\par\medskip
This product is divergent, it does not converge to a limiting value. Thankfully, the ascending sequence of natural odd numbers overshoots the worst-case scenario. According to this scenario we would not have contracted a single edge between two successive nodes.
