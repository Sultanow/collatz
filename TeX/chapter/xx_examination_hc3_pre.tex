\chapter{Examination of \boldmath$H_{C,3}$}

\section{The product in the condition for alpha's upper limit}
Let us take a closer look at the product contained in condition~\ref{eq:condition_max} for the case $k=3$. The exciting main question is, does this product have a limit value even in the worst case?

\par\medskip
Special sequences, for example the ascending sequence of odd integers $v_i=2i-1$ (beginning at $v_1=1$), allow us to transform this product into a limit analyzable function using the Pochhammer’s symbol (sometimes referred to as the \textit{rising factorial} or \textit{shifted factorial}), which is denoted by $(x)_n$ and defined as follows \cite{Ref_Zwillinger_Kokoska}, \cite[p.~679]{Ref_Brychkov} and \cite[p.~1005]{Ref_Trott}:
\[
(x)_n=x(x+1)(x+2)\cdots(x+n-1)=\prod_{i=0}^{n-1}(x+i)=\prod_{i=1}^{n}(x+i-1)=\frac{\Gamma(x+n)}{\Gamma(x)}
\]

Setting $v_i=2i-1$ into the product expressed by condition~\ref{eq:condition_max} and setting $x=\frac{k+1}{2k}$ into Pochhammer’s symbol $(x)_n$ interestingly makes it possible for us to perform the following transformation:
\[
\prod_{i=1}^{n}\left(1+\frac{1}{kv_i}\right)
=\frac{\prod_{i=1}^{n}(kv_i+1)}{\prod_{i=1}^{n}kv_i}
=\frac{\prod_{i=1}^{n}\left(k(2i-1)+1\right)}{k^n\prod_{i=1}^{n}(2i-1)}
=\frac{2^{2n}n!}{(2n)!}\cdot\frac{\Gamma\left(\frac{k+1+2kn}{2k}\right)}{\Gamma\left(\frac{k+1}{2k}\right)}
\]

\begin{example}
One simple example that is easy to recalculate may be provided by choosing $k=3$ and $n=4$:
\[
\left(1+\frac{1}{3*1}\right)\left(1+\frac{1}{3*3}\right)\left(1+\frac{1}{3*5}\right)\left(1+\frac{1}{3*7}\right)=1,6555=\frac{2^8*4!}{8!}\cdot\frac{\Gamma(\frac{14}{3})}{\Gamma(\frac{4}{6})}
\]
\end{example}

%The product in the numerator will be transformed into a form that allows us to use the Pochhammer’s symbol:
%\[\prod_{i=1}^{n}\left((2i-1)k+1\right)=2^nk^n\prod_{i=1}^{n}\frac{(2i-1)k+1}{2k}=2^nk^n\prod_{i=1}^{n}\frac{k+1+2ki-2k}{2k}=2^nk^n\prod_{i=1}^{n}\left(\frac{k+1}{2k}+i-1\right)\]

%The product can be written now as $2^nk^n(x)_n$, whwereby %$x=\frac{k+1}{2k}$:
%\[\prod_{i=1}^{n}\left((2i-1)k+1\right)=2^nk^n\frac{\Gamma\left(\frac{k+1+2kn}{2k}\right)}{\Gamma\left(\frac{k+1}{2k}\right)}\]

%The product in the denominator can be transformed as follows:
%\[\prod_{i=1}^{n}kv_i=k^n\prod_{i=1}^{n}v_i=k^n\prod_{i=1}^{n}(2i-1)=k^n\frac{(2n)!}{2^nn!}\]

\par\medskip
This product is divergent, it does not converge to a limiting value. Thankfully, the ascending sequence of natural odd numbers overshoots the worst-case scenario. According to this scenario we would not have contracted a single edge between two successive nodes. A worst case sequence $v_{n+1},v_n,\ldots,v_2,v_1$ describing a path in $H_{C,3}$ from $v_{n+1}$ down to $v_1$ allows at most one division by $2$ between two successive nodes. This sequence forms the following ascending continued fraction (cf. also \cite[p.~11]{Ref_Laarhoven}):

\begin{equation}
\label{eq:asc_continued_fraction}
	v_{n+1}=\cfrac{3\cfrac{3\cfrac{3v_1+1}{2}+1}{2}+1}{2}\dotsb
	=\frac{3^nv_1+\sum_{i=0}^{n-1}3^i2^{n-1-i}}{2^n}
	=\frac{3^n(v_1+1)-2^n}{2^n}
\end{equation}

\par\medskip
The sum of the products of the powers of three and two, contained within the above term, can be simplified to the difference $3^n-2^n$ by converting the sum expression into the form $(x-1)(1+x+x^2+\cdots+x^{n-2}+x^{n-1})=x^n-1$ as follows:
\[
\frac{2^n}{2^n}(3-2)\sum_{i=0}^{n-1}3^i2^{n-1-i}
=\frac{2^n}{\cancel{2^{n-1}}}\cdot\frac{3-2}{2}\sum_{i=0}^{n-1}3^i2^{\cancel{n-1}-i}
=2^n\left(\frac{3}{2}-1\right)\sum_{i=0}^{n-1}\left(\frac{3}{2}\right)^i
=2^n\left(\left(\frac{3}{2}\right)^n-1\right)
\]

\begin{example}
A concrete example for such a sequence is $v_1=31$, $v_2=47$, $v_3=71$, $v_4=107$, $v_5=161$. And, to follow that example, we can calculate the label of the vertex $v_5$ in a straightforward way:
\[
v_5=v_{n+1}=\frac{3^4(31+1)-2^4}{2^4}=161
\]
\end{example}

\begin{remark}
Ascending variants of a continued fraction, such as used in equation~\ref{eq:asc_continued_fraction}, shall not be confused with continued fractions as treated for example in \cite{Ref_Moore}, \cite{Ref_Hensley}, \cite{Ref_Borwe_etal}. These ascending continued fractions correspond to the so-called "Engel Expansions" \cite{Ref_Kraaikamp_Wu}.
\end{remark}

\par\noindent
As illustrated below, we can formulate the ascending continued fractions in a generalized fashion, whereas the equivalence to \ref{eq:asc_continued_fraction} is given by $b_1=b_2=b_3=b_4=2$ and $a_1=3^0$, $a_2=3^1$, $a_3=3^2$ and $a_4=3^3+3^4v_1$:
\[
\cfrac{a_1+\cfrac{a_2+\cfrac{a_3+\cfrac{a_4}{b_4}}{b_3}}{b_2}}{b_1}\dotsb=\frac{a_1}{b_1}+\frac{a_2}{b_1b_2}+\frac{a_3}{b_1b_2b_3}+\frac{a_4}{b_1b_2b_3b_4}+\cdots
\]

The generalized form of this ascending continued fraction is:
\begin{equation}
\label{eq:generalized_asc_continued_fraction}
v_{n+1}=\frac{k^n(kv_1-bv_1+1)-b^n}{b^n(k-b)}
\end{equation}

\par\medskip
This ascending continued fraction is helpful for examining the product in the condition~\ref{eq:condition_max} for alpha's upper limit. Setting this worst case sequence into the product expressed by condition~\ref{eq:condition_max}, we obtain a product that is limited, or to be more specific, which in the worst case $v_1=1$ converges (for $n$ to infinity) towards $2$:
\begin{equation}
\label{eq:product_simplification_k3}
\prod_{i=1}^{n}\left(1+\frac{1}{3v_{i}}\right)
=\prod_{i=1}^{n}\left(1+\frac{1}{3\frac{3^{i-1}(v_1+1)-2^{i-1}}{2^{i-1}}}\right)
=\prod_{i=1}^{n}\frac{3^i(v_1+1)-2^i}{3^i(v_1+1)-3*2^{i-1}}
=\frac{1}{v_1}-\frac{1}{v_1}\left(\frac{2}{3}\right)^n+1
\end{equation}

The above-illustrated last forming step, simplifies this product significantly into an expression waiving a product formulation. A detailed breakdown including all intermediate steps of this simplification is shown in the appendix~\ref{appx:product_simplification_k3}. The correctness of this simplification can be proven inductively too, which we detail in appendix~\ref{appx:proof_product_simplification_k3}. The most important and the most interesting aspect of this result is, that the above simplified term cannot exceed the value $2$, whatever you choose to insert into $n$ or into $v_1$:
\[
\frac{1}{v_1}-\frac{1}{v_1}\left(\frac{2}{3}\right)^{n+1}+1<2
\]

For this reason, the logarithmic product expression in the condition~\ref{eq:condition_max} cannot exceed the value one, strictly speaking the worst case for that condition is:
\[
n\log_23-\lfloor n\log_23\rfloor<2-1
\]

Thus, we have proved that for $k=3$ the condition~\ref{eq:condition_max} for alphas's upper limit is always met.

\section{Proving the worst case}
Let us compare two engel expansions. One we devide by $2^m$ afterwards. The other one we divide by $2$ after the penultimate step and we divide it by $2^{m-1}$ afterwards - in other words we perform one division by two a little bit earlier as we do it in the first engel expansion. One can immidiatly see the obvious inequality:
\[
\cfrac{1+\cfrac{3+\cfrac{3^2+\cfrac{3^3+3^4v_1}{2}}{2}}{2}}{2\cdot2^m}
<
\cfrac{1+\cfrac{3+\cfrac{3^2+\cfrac{3^3+3^4v_1}{2}}{2}}{2\cdot\textcolor{red}{\mathbf{2}}}}{2\cdot2^{m-1}}
\]

\[
\frac{1}{2\cdot2^m}+\cancel{\frac{3}{2^2\cdot2^m}+\frac{3^2}{2^3\cdot2^m}+\frac{3^3+3^4v_1}{2^4\cdot2^m}}
<
\frac{1}{2\cdot2^{m-1}}+\cancel{\frac{3}{2^2\cdot\textcolor{red}{\mathbf{2}}\cdot2^{m-1}}+\frac{3^2}{2^3\cdot\textcolor{red}{\mathbf{2}}\cdot2^{m-1}}\frac{3^3+3^4v_1}{2^4\cdot\textcolor{red}{\mathbf{2}}\cdot2^{m-1}}}
\]

Reversely, we obtain the maximum case by performing our divisions as early as possible. The result decreases, when we make a division by two later:
\[
\cfrac{1+\cfrac{3+\cfrac{3^2+\cfrac{3^3+3^4v_1}{2\cdot2^m}}{2}}{2}}{2}
>
\cfrac{1+\cfrac{3+\cfrac{3^2+\cfrac{3^3+3^4v_1}{2\cdot2^{m-1}}}{2\cdot\textcolor{red}{\mathbf{2}}}}{2}}{2}
\]

\[
\cancel{\frac{1}{2}+\frac{3}{2^2}}+\frac{3^2}{2^3}+\cancel{\frac{3^3+3^4v_1}{2^4\cdot2^m}}
>
\cancel{\frac{1}{2}+\frac{3}{2^2}}+\frac{3^2}{2^3\cdot\textcolor{red}{\mathbf{2}}}+\cancel{\frac{3^3+3^4v_1}{2^4\cdot\textcolor{red}{\mathbf{2}}\cdot2^{m-1}}}
\]

The difference between the maximum variant and the minimum variant is

\[
\frac{1}{2}+\frac{3}{2^2}+\frac{3^2}{2^3}+\frac{3^3+3^4v_1}{2^4\cdot2^m}-\frac{1}{2\cdot2^m}-\frac{3}{2^2\cdot2^m}-\frac{3^2}{2^3\cdot2^m}-\frac{3^3+3^4v_1}{2^4\cdot2^m}=\left(\frac{1}{2}+\frac{3}{2^2}+\frac{3^2}{2^3}\right)\left(1-\frac{1}{2^m}\right)=\left(\frac{3^3}{2^3}-1\right)\left(1-\frac{1}{2^m}\right)
\]

In order to consider the worst case of these permutaions we must substract the product, whereby n is the length of the continued fraction and m is the number of division we made afterwards:

\[
\left(\frac{3^{n-1}}{2^{n-1}}-1\right)\left(1-\frac{1}{2^m}\right)
\]

One simple example the chain we already know 31,47,71,107,161,242 (n=5) which we additionally one time divide by 2 (m=1):

\[
((31*3+1)/2*3+1)/2...=121
\]

Now we make permute the division by two, performing it very early and get: $(31*3+1)/2/2=23,5$ and $(23,5*3+1)/2=35,75$ and $(35,75*3+1)/2=54,125$ and $(54,125*3+1)/2=81,6875$ and finally $(81,6875*3+1)/2=123,03125$.

The difference is $123,03125-121=2,03125$ which we directly can calculate with our formula:
\[
\left(\frac{3^4}{2^4}-1\right)\left(1-\frac{1}{2^1}\right)=2,03125
\]
