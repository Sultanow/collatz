\chapter{Reachability of the Tree's Root Node}

\section{Determining the maximum alpha}
In the previous section we have shown how many divisions by two lead to a cycle in the Collatz tree. We now study the case in which a Collatz sequence reaches the root node $v_{n+1}=1$. Our proof builds on theorem~\ref{theo:1}. As in the last chapter we replace $(1+\frac{1}{3v_i})$ with the variable $\beta_i$:

\[
v_{n+1}=3^nv_1\prod_{i=1}^{n}\beta_i\prod_{i=1}^{n}2^{-\alpha_i}
\]

Setting $v_{n+1}=1$ leads to:

\begin{equation}
\label{eq:reach_1}
\begin{array}{l}
1=3^nv_1\prod_{i=1}^{n}\beta_i\prod_{i=1}^{n}2^{-\alpha_i}
\\[\medskipamount]
\prod_{i=1}^{n}2^{\alpha_i}=3^nv_1\prod_{i=1}^{n}\beta_i
\end{array}	
\end{equation}

\par\medskip
Equation~\ref{eq:reach_1} defines the maximum possible value of $\alpha$ for a given Collatz sequence. When a Collatz sequence reaches this alpha value, it finishes at the root node. The number of divisions by two required for this is referred to as $\hat\alpha$ subsequently:

\begin{equation*}
\begin{array}{l}
2^{\hat\alpha}=3^nv_1\prod_{i=1}^{n}\beta_i
\\[\medskipamount]
\hat\alpha=nlog_23+log_2v_1+log_2\prod_{i=1}^{n}\beta_i
\end{array}	
\end{equation*}

\par\medskip
In the previous chapter we proved $1<\prod_{i=1}^{n}\beta_i<2$. We use this knowledge to further restrict $\hat\alpha$ in theorem~\ref{theo:3}.

\bigskip
\begin{theorem}
\label{theo:3}
The maximum possible number of divisions by two in a Collatz sequence can be calculated as follows:
\[
\hat\alpha=\lfloor n\cdot log_23+log_2v_1\rfloor+1
\]
If a Collatz sequence reaches $\hat\alpha$, it ends with the result $v_{n+1}=1$.
\end{theorem}

\par\medskip
Since $\hat\alpha$ is a whole number, we truncate the fractional part. Knowing that $1<\prod_{i=1}^{n}\beta_i<2$ we add one to the result.

\bigskip
\begin{example}
Setting $v_{n+1}=13$ and $n=2$ leads to:
\[
v_{2+1}=3^2\cdot13\cdot\left(1+\frac{1}{3\cdot13}\right)\cdot\left(1+\frac{1}{3\cdot5}\right)\cdot2^{\lfloor2\cdot\log_23+log_213\rfloor+1}
\]
\end{example}

Building on $\hat\alpha$ we define the following restrictions on the alpha of a Collatz sequence:

\begin{equation}
\label{eq:reach_3}
n\le\alpha\le\hat\alpha
\end{equation}

Condition~\ref{eq:reach_3} is not only valid for $k=3$, but for all $k$. Similar to $\bar\alpha$, the variable $\hat\alpha$ could form the basis for a proof of the Collatz conjecture. As $\bar\alpha$ teaches us about cycles in the Collatz tree, $\hat\alpha$ leads us the way to its root node. If one shows that each Collatz sequence finally reaches $\hat\alpha$, the problem is solved as a whole. This is, however, not in the scope of this paper. It could be the foundation for a future work.
