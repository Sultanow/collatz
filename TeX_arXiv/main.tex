\documentclass[12pt]{amsart}
\usepackage{enumerate}
\usepackage[colorlinks=true, linkcolor=blue, urlcolor=blue, citecolor=blue, anchorcolor=blue, pdfborder={0 0 0}]{hyperref}
\usepackage{url}
\usepackage{graphicx,color}
\usepackage{cite}
\usepackage{amsthm, amsmath, amssymb, mathtools}
\usepackage[top=45truemm, bottom=45truemm, left=30truemm, right=30truemm]{geometry}
\usepackage{nicefrac}
\usepackage{cancel}

\newtheorem{theorem}{Theorem}[section]
\newtheorem{lemma}[theorem]{Lemma}
\newtheorem{corollary}[theorem]{Corollary}
\newtheorem{definition}[theorem]{Definition}
\newtheorem{proposition}[theorem]{Proposition}
\newtheorem{example}[theorem]{Example}
\theoremstyle{definition}
\newtheorem{remark}[theorem]{Remark}

\setlength{\skip\footins}{1.4pc plus 5pt minus 2pt}

\title[Engel Expansions in Collatz Sequences]{The Role of Engel Expansions in Collatz Sequences}

\author[F.\ Last1]{\href{https://orcid.org/0000-0000-0000-0000}{\includegraphics[scale=0.06]{orcid.png}}\hspace{1mm}First Last}
\address{First Lastname\\
Graduate School of Mathematics\\ XYZ University\\ City\\ Adresszusatz\\ ZIP\\ Germany}
\curraddr{}
\email{first.last@university.de}

\author[F.\ Last2]{First Last}
\address{First Lastname\\
Graduate School of Mathematics\\ XYZ University\\ City\\ Adresszusatz\\ ZIP\\ Germany}
\curraddr{}
\email{first.last@university.de}

\subjclass[2010]{37P99}
\keywords{Engel Expansions, Collatz Sequences}

\begin{document}
	
\begingroup
\let\MakeUppercase\relax
\maketitle
\endgroup

\begin{abstract}
The Collatz conjecture is a number theoretical problem, which has puzzled countless researchers using myriad approaches. Presently, there are scarcely investigations to treat the problem from the angle of the question "which are the corner cases the Collatz Sequences?". We pursue this question and to this end examine ascending continued fractions -- the so called Engel expansions. We demonstrate that Engel expansions form worst case sequences $v_1,v_2,\ldots,v_n,v_{n+1}$ maximizing $v_{n+1}$ and maximizing the product $(1+\nicefrac{1}{3v_1})(1+\nicefrac{1}{3v_2})\cdots(1+\nicefrac{1}{3v_n})(1+\nicefrac{1}{3v_{n+1}})$
\end{abstract}

\section{Introduction}
\label{introduction}

The Collatz conjecture is a well-known number theory problem and is the subject of numerous publications. An overview is provided by Lagarias \cite{Ref_Lagarias_2010}. Therefore, our description of the topic will be brief. The mathematician Lothar Collatz introduced a function $g:\mathbb{N}\rightarrow\mathbb{N}$ as follows:
\begin{equation}
\label{eq:func_collatz}
g(x)=
\begin{cases}
3x+1	&	2\nmid x\\
x/2		&	\text{otherwise}
\end{cases}
\end{equation}

In the following, we only consider compressed Collatz sequences that solely contain the odd members, such as described by Bruckman \cite{Ref_Bruckman_2008}, who used the more convenient function that opts out all even integers:
\begin{equation}
\label{eq:func_collatz_odd}
f(x)=(3x+1)\cdot2^{-\alpha(x)},\text{where}\hspace{1em}2^{\alpha(x)}\mathrel\Vert(3x+1)
\end{equation}

Note that $\alpha(x)$ is the largest possible exponent for which $2^{\alpha(x)}$ exactly divides $3x+1$. Especially for prime powers, one often says $p^\alpha$ \textit{divides} the integer $x$ \textit{exactly}, denoted as $p^\alpha\mathrel\Vert x$, if $p^\alpha$ is the greatest power of the prime $p$ that divides $x$.

\par\medskip
A (compressed) Collatz sequence $v_1,v_2,\ldots,v_n,v_{n+1}$ allowed at most one division by $2$ between two successive members. Dividing only once between two successive members, maximizes $v_{n+1}$. Such a sequence forms the following ascending continued fraction (cf. also \cite[p.~11]{Ref_Laarhoven}):

\begin{equation}
\label{eq:asc_continued_fraction}
v_{n+1}=\cfrac{3\cfrac{3\cfrac{3\cfrac{3v_1+1}{2}+1}{2}+1}{2}+1}{2}\dotsb
=\frac{3^nv_1+\sum_{i=0}^{n-1}3^i2^{n-1-i}}{2^n}
=\frac{3^n(v_1+1)-2^n}{2^n}
\end{equation}

\medskip
\begin{example}
A concrete example for such a sequence is $v_1=31$, $v_2=47$, $v_3=71$, $v_4=107$, $v_5=161$. And, to follow that example, we can calculate $v_5$ in a straightforward way:
\[
v_5=v_{n+1}=\frac{3^4(31+1)-2^4}{2^4}=161
\]

\par\medskip
Besides, by choosing a starting number $v_1=2^{n+1}-1$, we are able to infinitely generate sequences each forming an ascending continued fraction. As per equation~\ref{eq:asc_continued_fraction} the last member in this sequence is the odd number $v_{n+1}=3^n\cdot2-1$.
\end{example}

\bigskip
\begin{remark}
Ascending variants of a continued fraction, such as used in equation~\ref{eq:asc_continued_fraction}, shall not be confused with continued fractions as treated in \cite{Ref_Moore}, \cite{Ref_Hensley}, \cite{Ref_Borwe_etal}. Ascending continued fractions used in our case correspond to the so-called "Engel Expansions" \cite{Ref_Kraaikamp_Wu}.
\end{remark}

\par\noindent
As illustrated below, we can formulate the ascending continued fractions in a generalized fashion, whereas the analogy to \ref{eq:asc_continued_fraction} is given by $b_1=b_2=b_3=b_4=2$ and $a_1=3^0$, $a_2=3^1$, $a_3=3^2$ and $a_4=3^3+3^4v_1$:
\[
\cfrac{a_1+\cfrac{a_2+\cfrac{a_3+\cfrac{a_4}{b_4}}{b_3}}{b_2}}{b_1}\dotsb=\frac{a_1}{b_1}+\frac{a_2}{b_1b_2}+\frac{a_3}{b_1b_2b_3}+\frac{a_4}{b_1b_2b_3b_4}+\cdots
\]

\par\medskip
The generalized form of equation~\ref{eq:asc_continued_fraction} may be used to compute any of the above-named ascending continued fraction that has $a_i=k^{i-1}$, $b_i=b$ for $i\in\mathbb{N}$ and $a_n=k^{n-1}+k^nv_1$:

\begin{equation}
\label{eq:generalized_asc_continued_fraction}
v_{n+1}=\frac{k^n(kv_1-bv_1+1)-b^n}{b^n(k-b)}
\end{equation}

\newpage
\section{Include more divisions by two into an Engel expansion}
\label{sec:include_divisions_engel_expansion}
For calculating the largest possible $v_{n+1}$, we considered so far Engel expansions which contain only $n$ division by two within a Collatz sequence of $n+1$ memebers. In the following we include $m$ additional divisions by two and thus a total of $m+n$ divisions.

\par\bigskip
Let us take a look at two corner cases:
\begin{itemize}
	\item the one where we do the additional $m$ divisions by $2$ at the end and
	\item the one where we do these additional divisions at the very beginning.
\end{itemize}

\par\bigskip\noindent
\textbf{The first case} is our starting point to examine how the swapping a division by two affects the node $v_{n+1}$. For this, let us compare the Engel expansion where we devide by $2^m$ afterwards with one where we divide by $2$ in the penultimate step and by $2^{m-1}$ in last step. One can immediately recognize the following inequality with a mere look:

\[
\cfrac{1+\cfrac{3+\cfrac{3^2+\cfrac{3^3+3^4v_1}{2}}{2}}{2}}{2\cdot2^m}
<
\cfrac{1+\cfrac{3+\cfrac{3^2+\cfrac{3^3+3^4v_1}{2}}{2}}{2\cdot\textcolor{red}{\mathbf{2}}}}{2\cdot2^{m-1}}
\]

\par\bigskip
To put it simply, in the expansion on the right side of the above-shown inequality we perform one division by two a little bit earlier as we do it in the expansion on the left side of the expansion. Almost all summands of both expansions cancel out each other:

\[
\frac{1}{2\cdot2^m}+\cancel{\frac{3}{2^2\cdot2^m}+\frac{3^2}{2^3\cdot2^m}+\frac{3^3+3^4v_1}{2^4\cdot2^m}}
<
\frac{1}{2\cdot2^{m-1}}+\cancel{\frac{3}{2^2\cdot\textcolor{red}{\mathbf{2}}\cdot2^{m-1}}+\frac{3^2}{2^3\cdot\textcolor{red}{\mathbf{2}}\cdot2^{m-1}}\frac{3^3+3^4v_1}{2^4\cdot\textcolor{red}{\mathbf{2}}\cdot2^{m-1}}}
\]

\par\bigskip\noindent
\textbf{The second case} deals with Engel expansions where we perform that additional $m$ divisions by two as early as possible. The resulting value decreases, when we make a division by two later:
\[
\cfrac{1+\cfrac{3+\cfrac{3^2+\cfrac{3^3+3^4v_1}{2\cdot2^{m-1}}}{2\cdot\textcolor{red}{\mathbf{2}}}}{2}}{2}
<
\cfrac{1+\cfrac{3+\cfrac{3^2+\cfrac{3^3+3^4v_1}{2\cdot2^m}}{2}}{2}}{2}
\]

\par\bigskip
Also here almost all summands of both Engel expansions, they cancel each other out:

\[
\cancel{\frac{1}{2}+\frac{3}{2^2}}+\frac{3^2}{2^3\cdot\textcolor{red}{\mathbf{2}}}+\cancel{\frac{3^3+3^4v_1}{2^4\cdot\textcolor{red}{\mathbf{2}}\cdot2^{m-1}}}
<
\cancel{\frac{1}{2}+\frac{3}{2^2}}+\frac{3^2}{2^3}+\cancel{\frac{3^3+3^4v_1}{2^4\cdot2^m}}
\]

\par\medskip
While the first case minimizes the value of the node $v_{n+1}$, the second case maximizes it. The difference between the maximum and the minimum is given by the following equation:
\[
\frac{3^{n-1}\left(\frac{3v_1+1}{2\cdot2^m}+1\right)-2^{n-1}}{2^{n-1}}-\frac{3^n\left(v_1+1\right)-2^n}{2^{n+m}}=\left(\frac{3^{n-1}}{2^{n-1}}-1\right)\left(1-\frac{1}{2^m}\right)
\]
%\begin{flalign*}
%&\frac{3^{n-1}\left(\frac{3v_1+1}{2\cdot2^m}+1\right)-2^{n-1}}{2^{n-1}}-\frac{3^n\left(v_1+1\right)-2^n}{2^{n+m}}\\
%=&\frac{3^{n-1}\cdot\left(3v_1+1+2^{m+1}\right)-2^{n-1}\cdot2^{m+1}-3^n\left(v_1+1\right)+2^n}{2^{m+1}\cdot2^{n-1}}\\
%=&\frac{3^{n-1}+3^{n-1}\cdot2^{m+1}-2^{n+m}-3^n+2^n}{2^{n+m}}=\frac{3^{n-1}-3\cdot3^{n-1}+3^{n-1}\cdot2^{m+1}-2^{n+m}+2^n}{2^{n+m}}\\
%=&\frac{-2\cdot3^{n-1}+3^{n-1}\cdot2^{m+1}-2^{n+m}+2^n}{2^{n+m}}=\frac{\left(2\cdot3^{n-1}-2^n\right)\left(2^m-1\right)}{2^n\cdot2^m}\\
%=&\left(\frac{3^{n-1}}{2^{n-1}}-1\right)\left(1-\frac{1}{2^m}\right)
%\end{flalign*}

\par\medskip
This has the consequence that for a given sequence consisting of $n+1$ members, between which a total of $n+m$ divisions have taken place, the permutation of these divisions has a very limited effect on the node $v_{n+1}$ as described by theorem~\ref{theo:permutation}.

\par\medskip
\begin{theorem}
	\label{theo:permutation}
	Let $v_1,v_2,\ldots,v_n,v_{n+1}$ be a sequence in which a total of $n+m$ divisions by two took place. No matter how these divisions are permuted, i.e. performed sooner or later, the last member $v_{n+1}$ can differ at most by the following product:
	\[
	\left(\frac{3^{n-1}}{2^{n-1}}-1\right)\left(1-\frac{1}{2^m}\right)
	\]
\end{theorem}

\section{The product of reciprocated Collatz members incremented by one}
Let us take a closer look at the product $(1+\nicefrac{1}{3v_1})(1+\nicefrac{1}{3v_2})\cdots(1+\nicefrac{1}{3v_n})(1+\nicefrac{1}{3v_{n+1}})$ that we mentioned at the beginning and use the ascending continued fractions for examining it. The exciting question is: Does this product have a limit value even in the case where between successive Collatz sequence members $v_i$ and $v_{i+1}$ a division by two has been performed only once? Using the Engel expansion (that sequence maximizing $v_{n+1}$) for calculating this product, we obtain a product which is limited, or to be more specific, we obtain a product that cannot exceed the value $\nicefrac{4}{3}$:
\begin{equation}
\label{eq:product_simplification_k3}
\prod_{i=1}^{n}\left(1+\frac{1}{3v_{i}}\right)
=\prod_{i=1}^{n}\frac{3^i(v_1+1)-2^i}{3^i(v_1+1)-3\cdot2^{i-1}}
=\frac{1}{v_1}-\frac{1}{v_1}\left(\frac{2}{3}\right)^n+1
\end{equation}

The largest possible value of this product can be received by setting $v_1=n=1$. In this case the product returns $1-\nicefrac{2}{3}+1$ and the value of $n$ cannot be larger than one, since we already reached the end of the Collatz sequence. Inserting any odd value greater than one into $v_1$ leads to a smaller product, no matter how large you choose $n$. When setting $v_1=3$ the product converges (for $n$ to infinity) from below towards $\nicefrac{4}{3}$. The larger you choose $v_1$, the smaller becomes the product.

\section{Include additional divisions into the product}
How does this product looks like if we include the additional $m$ divisions into the Engel expansion as per section~\ref{sec:include_divisions_engel_expansion}? To answer this question, we consider the sequence $v_1,v_2,\ldots,v_n,v_{n+1}$ and we set $v_2=\nicefrac{3v_1+1}{2\cdot2^{m}}$. Then reusing the continued fraction given by equation~\ref{eq:asc_continued_fraction}, we obtain:

{\setlength{\jot}{1.2em}
	\begin{flalign}
	\label{eq:asc_continued_fraction_m}
	v_{n+1}&=\cfrac{3\cfrac{3\cfrac{3\cfrac{3v_1+1}{2\cdot2^m}+1}{2}+1}{2}+1}{2}\dotsb
	=\cfrac{3\cfrac{3\cfrac{3v_2+1}{2}+1}{2}+1}{2}\dotsb
	=\frac{3^{n-1}(v_2+1)-2^{n-1}}{2^{n-1}}\\
	\notag
	&=\frac{3^{n-1}(\frac{3v_1+1}{2\cdot2^{m}}+1)-2^{n-1}}{2^{n-1}}=\frac{3^nv_1+3^{n-1}+3^{n-1}2^{m+1}}{2^{m+n}}-1
	\end{flalign}}

\par\noindent
The product will be calculated by using equation~\ref{eq:product_simplification_k3}:
\begin{flalign}
\label{eq:product_k3_m}
\prod_{i=1}^{n}\left(1+\frac{1}{3v_{i}}\right)&=\left(1+\frac{1}{3v_1}\right)\cdot\prod_{i=2}^{n}\left(1+\frac{1}{3v_{i}}\right)\\
\notag
&=\left(1+\frac{1}{3v_1}\right)\cdot\prod_{i=1}^{n-1}\left(1+\frac{1}{3v_{i+1}}\right)\\
\notag
&=\left(1+\frac{1}{3v_1}\right)\cdot\left(\frac{1}{v_2}-\frac{1}{v_2}\left(\frac{2}{3}\right)^{n-1}+1\right)
\end{flalign}

\par\noindent
Finally substituting $v_2=\nicefrac{3v_1+1}{2\cdot2^{m}}$ into equation~\ref{eq:product_k3_m} leads to the simplified formula of the product:
\begin{equation}
\label{eq:product_simplification_k3_m}
\prod_{i=1}^{n}\left(1+\frac{1}{3v_{i}}\right)=\left(1+\frac{1}{3v_1}\right)\cdot\frac{1-\left(\frac{2}{3}\right)^{n-1}+v_2}{v_2}=\frac{1+2^{m+1}}{3v_1}-\frac{2^m}{v_1}\left(\frac{2}{3}\right)^n+1
\end{equation}

\medskip
\begin{example}
An example provides the sequence $v_1=661$, $v_2=31$, $v_3=47$, $v_4=71$, $v_5=107$. When we input $v_1=661$ with $m=5$ and $n=4$ into equation~\ref{eq:asc_continued_fraction_m} we retrieve the value of $v_5$:
\[
v_5=v_{n+1}=\frac{3^4\cdot661+3^3+3^3\cdot2^6}{2^9}-1=107
\]
In this sequence five $(m=5)$ additional divisions by two took place in the first step using $v_1=661$:
\[
\frac{3\cdot661-1}{2\cdot2^5}=v_2=31
\]

\par\medskip\noindent
We now verify the formula for the product by taking this particular example. To this end we input $v_1=661$ together with $m=5$ and $n=4$ into equation~\ref{eq:product_simplification_k3_m}:
\begin{flalign*}
\left(1+\frac{1}{3\cdot661}\right)\left(1+\frac{1}{3\cdot31}\right)\left(1+\frac{1}{3\cdot47}\right)\left(1+\frac{1}{3\cdot71}\right)&=\frac{1+2^{6}}{3\cdot661}-\frac{2^5}{661}\left(\frac{2}{3}\right)^4+1\\
&=1.023215853
\end{flalign*}
\end{example}

\section{Condition for a limited growth of the Engel expansion}
\label{sec:condition_limited_growth}
Let us look now into the question of what condition must be met to prevent a greater growth than a decline in Collatz sequences. Specifically we consider an Engel expansion comprising $n+1$ sequence members that include $m$ additional divisions by two at the beginning. The last member $v_{n+1}$ in such a sequence can be calulated by formula~\ref{eq:asc_continued_fraction_m}. In order to restrict the growth of this sequence, we require that the last member has to be smaller than the first one. For this we define the condition $v_{n+1}<v_1$:

\[
\frac{3^nv_1+3^{n-1}+3^{n-1}2^{m+1}}{2^{m+n}}-1<v_1
\]

\par\medskip\noindent
Reshaping this inequality leads to the following condition:

\begin{equation}
\label{eq:condition_limited_growth}
\frac{3^{n-1}\left(2^{m+1}-2\right)}{2^{m+n}-3^n}-1<v_1
\end{equation}

\section{Engel expansions maximize the product}
The question which sequence maximizes the target node $v_{n+1}$ ties into the question: Which sequence maximizes the product? The product formula that do not depend from all vertices $v_1,v_2,\ldots,v_n$ depends only from $2^\alpha$, from the starting node $v_1$ and the target node $v_{n+1}$:
\[
\prod_{i=1}^{n}\left(1+\frac{1}{kv_i}\right)=\frac{2^{\alpha_1+\ldots+\alpha_n}v_{n+1}}{k^nv_1}
\]

\par\medskip
In order to maximize this product, one needs to maximize the target node $v_{n+1}$, which exactly the Engel expansion does. Hence, for a given $v_1$, the Engel expansion is the worst case sequence maximizing the product of reciprocated Collatz members incremented by one $(1+\nicefrac{1}{3v_1})(1+\nicefrac{1}{3v_2})\cdots(1+\nicefrac{1}{3v_n})(1+\nicefrac{1}{3v_{n+1}})$.

\bibliographystyle{unsrt}
\bibliography{references}

\end{document}