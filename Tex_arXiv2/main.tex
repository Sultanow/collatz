\documentclass[12pt]{amsart}
\usepackage{enumerate}
\usepackage[colorlinks=true, linkcolor=blue, urlcolor=blue, citecolor=blue, anchorcolor=blue, pdfborder={0 0 0}]{hyperref}
\usepackage{url}
\usepackage{graphicx,color}
\usepackage{cite}
\usepackage{amsthm, amsmath, amssymb}
\usepackage{mathtools}
\usepackage[top=45truemm, bottom=45truemm, left=30truemm, right=30truemm]{geometry}
\usepackage{nicefrac}
\usepackage{cancel}
\usepackage{float}
\usepackage{tabularx}
\usepackage{makecell}
\usepackage{array}
\usepackage{ragged2e}

\newcolumntype{P}[1]{>{\RaggedRight\hspace{0pt}}p{#1}}

\newcolumntype{L}{>{\begin{math}}l<{\end{math}}}%
\newcolumntype{C}{>{\begin{math}}c<{\end{math}}}%
\newcolumntype{R}{>{\begin{math}}r<{\end{math}}}%

\newtheorem{theorem}{Theorem}[section]
\newtheorem{lemma}[theorem]{Lemma}
\newtheorem{corollary}[theorem]{Corollary}
\newtheorem{definition}[theorem]{Definition}
\newtheorem{proposition}[theorem]{Proposition}
\newtheorem{example}[theorem]{Example}
\theoremstyle{definition}
\newtheorem{remark}[theorem]{Remark}

\setlength{\skip\footins}{1.4pc plus 5pt minus 2pt}

\title[Binary p-adic integers in Collatz Sequences]{Binary p-adic integers in Collatz sequences}

\author[F.\ Last1]{\href{https://orcid.org/0000-0000-0000-0000}{\includegraphics[scale=0.06]{orcid.png}}\hspace{1mm}First Last}
\address{First Lastname\\
Graduate School of Mathematics\\ XYZ University\\ City\\ Adresszusatz\\ ZIP\\ Germany}
\curraddr{}
\email{first.last@university.de}

\author[F.\ Last2]{First Last}
\address{First Lastname\\
Graduate School of Mathematics\\ XYZ University\\ City\\ Adresszusatz\\ ZIP\\ Germany}
\curraddr{}
\email{first.last@university.de}

\subjclass[2010]{37P99}
\keywords{2-adic numbers, binary residue system}

\begin{document}
	
\begingroup
\let\MakeUppercase\relax
\maketitle
\endgroup

\begin{abstract}
The Collatz conjecture is a number theoretical problem, which has puzzled countless researchers using myriad approaches. We describe an approach from the perspective of 2-adic (binary) algebra.
\end{abstract}

\newpage
{\renewcommand{\arraystretch}{1.8}
\begin{table}[H]
	\centering
	\begin{tabular}{|P{1.4cm} p{13.4cm}|}
		\hline
		\multicolumn{2}{|l|}{\thead[l]{\textbf{Fundamentals short and sweet}}}
		\\
		unit & An element $a$ of a ring $R$ is called a "unit" (an invertible element) if there exist an element $b$ such that $ab=1$ \cite[p.~24]{Ref_Northcott_1953}. Units are elements with inverses with respect to multiplication in the ring. Let $F$ be a field, then an element $a$ of $F$ is a non-unit iff $a=0$. The sum of any two non-units in $F$ is again a non-unit in $F$.
		\\
		unitary ring & A unitary ring is a ring with a multiplicative identity $1$ (which differs from the additive identity $1\ne0$) such that $1a=a=a1$ for all elements $a$ of the ring.
		\\
		Ideal & Let $(R,+,\cdot)$ be a commutative unitary ring. Then the subset $I\subseteq R$ is called an ideal of $R$ if $(I,+)$ is a commutative group and if $xI\subseteq I$ for all $x\in R$, see \cite[p.~66-67]{Ref_Wolfart_2011}.
		\\
		quot. ring & Using an ideal of a ring $I\subseteq R$, we may define an equivalence relation $\sim$ on $R$ by $a\sim b$ iff $a-b$ is in $I$ \cite[p.~69]{Ref_Schulze-Pillot_2015}. The equivalence class of $a$ in $R$ is given by $[a]=a+I:=\{a+r|r\in I\}$ for $r\in R$ and referred to as "residue class of $a$ modulo $I$", see \cite[p.~120]{Ref_Schubert_2009}, \cite[p.~70]{Ref_Schulze-Pillot_2015}. The set of all these equivalence classes becomes the quotient ring (residue class ring) modulo the ideal $I$, denoted by $R/I$.
		\\
		compl. residue system & Let $I\subseteq R$ be an ideal and $[a]$ the residue classes of $a$ modulo $I$, which means that $a+I=b+I$ when $a\equiv b\mod I$ or respectively $a-b\in I$ \cite[p.~70]{Ref_Schulze-Pillot_2015}. $R$ is the disjoint union of the different residue classes $a$ modulo $I$. A subset $M\subseteq R$, which contains exactly one element from each of these residue classes, is called a complete residue system of $R$ modulo $I$, see \cite[p.~70]{Ref_Schulze-Pillot_2015}.
		\\
		$[a]_n$ & The residue class (also termed congruence class) of the integers for a modulus $n$ is the set $[a]_n=\{a+kn|k\in\mathbb{Z}\}$ and sometimes denoted by $\bar a_n$ or by $a+n\mathbb{Z}$, see \cite[p.~15]{Ref_Wolfart_2011}, \cite[p.~120]{Ref_Schubert_2009}, \cite[p.~25]{Ref_Mueller-Stach_2011}.
		\\
		$\mathbb{Z}/n\mathbb{Z}$ & The set of all residue classes $[a]_n$ is called the ring of integers modulo $n$ and denoted by $\mathbb{Z}/n\mathbb{Z}=\{[a]_n|a\in\mathbb{Z}\}$ and trivially $\mathbb{Z}/0\mathbb{Z}=\mathbb{Z}$ and for all $n\ne0$ we have $\mathbb{Z}/n\mathbb{Z}=\{[0],[1],\ldots,[n-1]\}$, see \cite[p.~15]{Ref_Wolfart_2011}, \cite[p.~25]{Ref_Mueller-Stach_2011}. 
		\\ \hline
	\end{tabular}
\end{table}}

{\renewcommand{\arraystretch}{1.8}
\begin{table}[H]
	\centering
	\begin{tabular}{|P{1.4cm} p{13.4cm}|}
		\hline
		direct prod. & If $R_1,R_2,\ldots,R_n$ are rings, the cartesian product $R_1\times R_2\times\ldots\times R_n$ forms the set of all ordered $n$-tuples $(r_1,r_2,\ldots,r_n)$, where $r_i\in R_i$. The addition and multiplication of these n-tuples is defined "coordinatewise" by components. The resulting ring is called a "direct product" of the original rings $R_i$ \cite[p.~51]{Ref_Wolfart_2011}, \cite[p.~169]{Ref_Fraleigh_2014}.
		\\
		princip. ideal & A "principle ideal" is an ideal in a ring $R$ which is generated by a single element $a$ of $R$ through multiplication by every element of $R$. There are some rings in which every ideal is a principle ideal, so-called "principle ideal rings" \cite[p.~68]{Ref_Wolfart_2011}.
		\\
		max. ideal & A proper Ideal $M$ of a ring $R$ is called "maximal ideal" of $R$  if there is no other proper ideal $N$ of $R$ properly containing $M$ \cite[p.~247]{Ref_Fraleigh_2014}, \cite[p.~37]{Ref_Northcott_1953}. A Note on "proper containment": If $R$ is any set, then $R$ is the improper subset of $R$. Any other subset $N\ne R$ is a proper subset of $R$ and denoted by $N\subset R$ or $N\varsubsetneq R$ \cite[p.~2]{Ref_Fraleigh_2014}.
		\\
		prime ideal & Let $a$ and $b$ are two elements of $R$ and $P$ a proper ideal such that their product $ab$ is an element of $P$. $P$ is called a prime ideal if at least one of $a$ and $b$ belongs to $P$, in other words from $ab\in P$ and $a\notin P$ always follows $b\in P$ \cite[p.~9]{Ref_Northcott_1953}.
		\\
		max. prime ideal & A proper prime ideal $P$ is said to be a "maximal prime ideal" of the ring $R$, if there is no other proper prime ideal containing $P$ \cite[p.~23]{Ref_Northcott_1953}.
		\\
		local ring & A commutative ring $R$ is called a local ring if it has a unique maximal ideal $M$ \cite[p.~522]{Ref_Rotman_2005}.
		\\
		Noeth. ring & A ring $R$ is called "Noetherian" when in $R$ the maximal condition for ideals is satisfied, in other words if every ideal $I$ of $R$ is finitely generated, that is, if we can find a finite set $a_1,a_2,\ldots,a_n$ of elements, such that $I=Ra_1+Ra_2+\ldots+Ra_n$ \cite[p.~19, 101]{Ref_Northcott_1953}.
		\\
		semi-local ring & A semi-local ring is a Noetherian ring which has only a finite number of maximal ideals \cite[p.~107]{Ref_Northcott_1953}.
		\\ \hline
	\end{tabular}
\end{table}}

{\renewcommand{\arraystretch}{1.8}
\begin{table}[H]
	\centering
	\begin{tabular}{|P{1.4cm} p{13.4cm}|}
		\hline
		zero seq. & A zero sequence is a sequence, which converges towards $0$ \cite[p.~154]{Ref_Schmidt_2007}. Given the context of ideal theory, let $R$ be a ring and $I$ an ideal. In the ring $R^\mathbb{N}=\prod_{n\in\mathbb{N}}R$, which is the repeated direct product of $R$ with itself, a sequence $(x_i)_{i\in\mathbb{N}}$ is called a zero sequence if for every $s\in\mathbb{N}$ there exist a $N\in\mathbb{N}$ (depending on $s$) such that $x_n\in I^s$ for all $n>N$.
		\\
		Cauchy seq. in $\mathbb{Q}$, $\mathbb{R}$ & A sequence $(x_i)_{i\in\mathbb{N}}$ in $\mathbb{Q}$ or $\mathbb{R}$ is a Cauchy sequence if for any $\epsilon>0$ there exists a positive integer $N$ such that $|x_n-x_m|<\epsilon$ for all $n,m\ge N$, see \cite[p.~153]{Ref_Schmidt_2007}, \cite[p.~24]{Ref_Higham_2015},\cite[p.~10]{Ref_Koblitz_1984}.
		\\
		Cauchy seq. in a ring & Let $(x_i)_{i\in\mathbb{N}}$ be a sequence of elements in $R^\mathbb{N}$, the repeated direct product of a ring with itself, and $I$ an ideal in $R$. This sequence is a Cauchy sequence if for every $s\in\mathbb{N}$ there exist a $N\in\mathbb{N}$ such that $x_n-x_m\in I^s$ for all $n,m>N$.
		\\
		Cauchy seq. in a local ring & Let $(x_i)_{i\in\mathbb{N}}$ be a sequence of elements in a local ring $R$ and $M$ is the maximal ideal of $R$. This sequence is a Cauchy sequence if, given any $s\in\mathbb{N}$, we can always find an integer $N$ such that $x_n-x_m\in M^s$ whenever $n>m>N$, see \cite[p.~63, 85]{Ref_Northcott_1953}. It is a Cauchy sequence iff $x_n-x_{n-1}\rightarrow0$ as $n\rightarrow\infty$ \cite[p.~85]{Ref_Northcott_1953}.
		\\
		compl. of a ring & Let $R$ be a ring, $I$ an indeal, $I_{ZS}$ the ideal of all zero sequences in $R^\mathbb{N}$, and $S_{CS}$ the subring of $R^\mathbb{N}$ containing all Cauchy sequences. The quotient ring $\hat R_I:=S_{CS}/I_{ZS}$ is called the completion of $R$ with respect to $I$. $S_{CS}/I_{ZS}$ is the residue class ring of $S_{CS}$ modulo $I$.
		\\
		concor. ext. & Let $R,S$ be local rings. If a sequence of elements of $S$ is a Cauchy sequence in $S$ iff it is a Cauchy sequence in $R$, then we say that $R$ is a "concordant extension" of $S$ \cite[p.~87]{Ref_Northcott_1953}. When $R,S$ are semi-local rings $R\subseteq S$, $R$ is said to be a "concordant extension" of $S$ if a sequence $(s_n)$ of elements in $S$ is regular in $S$ iff $(s_n)$ is regular in $R$ \cite{Ref_Batho_1959}.
		\\
		compl. of a local ring & Let $S$ be a local ring. A local ring $R$ will be called a completion of $S$ if $R$ is a concordant extension of $S$ and $R$ is complete and if every element of $R$ is the limit of a sequence of elements of $S$. Each local ring has a completion \cite[p.~92]{Ref_Northcott_1953}.	 
		\\
		compl. local ring & A local ring $R$ is called "complete" if every Cauchy sequence composed of elements of $R$ has a limit in $R$ \cite[p.~85]{Ref_Northcott_1953}, \cite[p.~184]{Ref_Kemper_2011}.
		\\ \hline
	\end{tabular}
\end{table}}

{\renewcommand{\arraystretch}{1.8}
\begin{table}[H]
	\centering
	\begin{tabular}{|P{1.4cm} p{13.4cm}|}
		\hline
		$p$-adic val. for $\mathbb{Z}$ & Fix a prime number $p$ in $\mathbb{Z}$. The $p$-adic valuation of a nonzero integer $n=r\cdot p^{v_p(n)}$ is the highest exponent $v_p(n)$ such that $p^{v_p(n)}$ divides $n$ (we say $p^{v_p(n)}$ divides $n$ "exactly"). Hence $p$ and $r$ are coprime. If $n$, $p$ are coprime then $v_p(n)=0$, and by convention $v_p(0)=\infty$, see \cite{Ref_Herwig_2011}.
		%https://de.wikipedia.org/wiki/Bewertung_(Algebra):
		%Tritt eine Primzahl p nicht in der Primfaktorzerlegung von n auf, dann ist v_p(n)=0
		\\
		$p$-adic val. for $\mathbb{Q}$ & The $p$-adic valuation can be extended to the field of rational numbers. Let $x=n\cdot s^{-1}$ be a rational number, then $v_p(x)=v_p(n)-v_p(s)$. Any nonzero rational number $x$ can be uniquely represented as $x=rp^{v_p(x)}s^{-1}$, where $r,s\in\mathbb{Z}$, $s>0$, and $\gcd(r,s)=\gcd(r,p)=\gcd(s,p)=1$, see \cite[p.~154]{Ref_Schmidt_2007}, \cite{Ref_Weisstein_1}.
		\\
		$p$-adic norm & Let $x$ be any number in $\mathbb{Q}$, for which we already know that it can be written as $x=rp^{v_p(x)}s^{-1}$, where $p$ is a prime number, $s>0$ and $r$ are integers not divisible by $p$. The $p$-adic norm of $x$ is defined by $|x|_p=p^{-v_p(x)}$ for $x\ne0$, and $|0|_p=0$, see \cite{Ref_Herwig_2011}, \cite[p.~154]{Ref_Schmidt_2007}, \cite{Ref_Weisstein_2}.
		\\
		$p$-adic dist. &  Let $x,y\in\mathbb{Q}$. The $p$-adic distance between $x$ and $y$ is defined by $d_p(x,y)=|x-y|_p$, see \cite[p.~155]{Ref_Schmidt_2007}.
		\\
		$\mathbb{Q}_p$ & The field $\mathbb{Q}_p$ of $p$-adic numbers is the set of equivalence classes of Cauchy sequences \cite[p.~10]{Ref_Koblitz_1984}. The elements of $\mathbb{Q}_p$, the so-called $p$-adic numbers, are eqivalence classes of Cauchy sequences $(a_n)_{n\in\mathbb{N}}$ in $\mathbb{Q}$ with respect to the equivalence relation $(a_n)\sim(b_n)$ if $(a_n-b_n)$ is a $p$-adic zero sequence, see \cite[p.~159]{Ref_Schmidt_2007}. Furthermore $\mathbb{Q}_p$ is the completion of $\mathbb{Q}$ with respect to the $p$-adic distance $d_p$ \cite[p.~159]{Ref_Schmidt_2007}. 
		\\
		$\mathbb{Z}_p$ & The ring $\mathbb{Z}_p$ of $p$-adic integers is the completion of $\mathbb{Z}$ with respect to the $p$-adic norm. That is, $\mathbb{Z}_p$ is the set of all equivalence classes of Cauchy sequences $(a_n)$ where $(a_n)$ and $(b_n)$ are equivalent if $\lim_{n\to\infty}|a_n-b_n|_p=0$, see \cite{Ref_Gupta_2018}. $\mathbb{Z}_p$ is a local ring whose maximal ideal is the principal ideal $p\mathbb{Z}_p=\{x\in\mathbb{Q}_p:|x|_p<1\}$, see \cite[p.~74]{Ref_Gouvea_2020}.
		\\ \hline
	\end{tabular}
\end{table}}

%\section{Introduction}
%\label{introduction}

%tbd
For $k=1$, let's say that $\frac{v_1 \beta}{2^{\alpha}}=\frac{v_1 + \delta}{2^{\alpha}}=1$ it is clear that an overflow is provoked by $\delta$ the accumulation of "+1", and it occurs before this $\delta$ reaches $v_1$ since $v_1$ is already larger than half of its next power of 2 (the one it will overflow to). So it is clear that $\delta<v_1$ and therefore $\beta<2$

For $k=3$, however, the multiplication by 3 make it possible that $\delta$ grows larger than $v_1$

Let's say $\frac{3^nv_1 \beta}{2^{\alpha}}=\frac{3^nv_1 + \delta}{2^{\alpha}}=1$

Imagine we are at an intermediate step $v_i=17$ (10001) and we already have some "+1 accumulation" $\delta=13$ (1101). $\delta<v_i$, the sum (30 or 11110) still bellow overflow point (power of 2 just above $17$ which is $32$) and in the case of $k=1$, $\delta$ would grow up to overflow with the guarantee it will stay smaller tan $v_1$ since the overflow point don't move.

With the $k=3$ case, the overflow point can move higher than the next power of 2 above 17 (due to multiplication by 3). If you multiply by 3, you get $v_{i+1}=51$ (110011) and $\delta=39$ (100111), but as you can see, the overflow point is not above the main term anymore (the sum is already larger than that power of 2), and does not prevent $\delta$ to grow larger than the $v_i$'s with accumulated "+1". In which case you can end up with $\beta>2$ and therefore 3.8 would not be true anymore.
\\
Ok, we proove our formula for alpha inductively.

\[
\hat{\alpha}(n)=\lfloor n\cdot\log_23+\log_2 v_1\rfloor+1
\]

For the base case $n=1$ we have an always true statement:

\[
\hat{\alpha}(1)=\lfloor \log_23+\log_2 v_1\rfloor+1
\]

Now we need to show that an arbitary $n$ induces an always a true statement for $n+1$:

\[
\hat{\alpha}(n+1)=\lfloor (n+1)\cdot\log_23+\log_2 v_1\rfloor+1
\]

In order to show that this statement is true, we set

\[
\frac{2^{\hat\alpha(n+1)}}{2^{\hat\alpha(n)}}=\frac{2^{\lfloor (n+1)\cdot\log_23+\log_2 v_1\rfloor}}{2^{\lfloor n\cdot\log_23+\log_2 v_1\rfloor}}=\begin{cases}
2& \text{if } n~\text{even}\\
4& \text{otherwise}
\end{cases}
\]

This finally means that $\hat{\alpha}(n+1)=2*\hat{\alpha}(n)$ for all even $n$ and $\hat{\alpha}(n+1)=4*\hat{\alpha}(n)$ for all odd $n$.

\vspace{1em}
\bibliographystyle{unsrt}
\bibliography{references}

\end{document}