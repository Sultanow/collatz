\documentclass[12pt]{amsart}
\usepackage{enumerate}
\usepackage{enumitem}
\usepackage[colorlinks=true, linkcolor=blue, urlcolor=blue, citecolor=blue, anchorcolor=blue, pdfborder={0 0 0}]{hyperref}
\usepackage{url}
\usepackage{graphicx,color}
\usepackage{cite}
\usepackage{amsthm, amsmath, amssymb}
\usepackage{mathtools}
\usepackage[top=45truemm, bottom=45truemm, left=30truemm, right=30truemm]{geometry}
\usepackage{nicefrac}
\usepackage{cancel}
\usepackage{float}
\usepackage{tabularx}
\usepackage{makecell}
\usepackage{multirow}
\usepackage{array}
\usepackage{ragged2e}

\newcolumntype{P}[1]{>{\RaggedRight\hspace{0pt}}p{#1}}

\newcolumntype{L}{>{\begin{math}}l<{\end{math}}}%
\newcolumntype{C}{>{\begin{math}}c<{\end{math}}}%
\newcolumntype{R}{>{\begin{math}}r<{\end{math}}}%

\newtheorem{theorem}{Theorem}[section]
\newtheorem{lemma}[theorem]{Lemma}
\newtheorem{corollary}[theorem]{Corollary}
\newtheorem{definition}[theorem]{Definition}
\newtheorem{proposition}[theorem]{Proposition}
\newtheorem{example}[theorem]{Example}
\theoremstyle{definition}
\newtheorem{remark}[theorem]{Remark}

\setlength{\headsep}{2em}
\setlength{\skip\footins}{1.4pc plus 5pt minus 2pt}

\title[Cycles in kx+c functions]{Cycles in $kx+c$ functions}

\author[F.\ Last1]{\href{https://orcid.org/0000-0000-0000-0000}{\includegraphics[scale=0.06]{orcid.png}}\hspace{1mm}Darrell Cox}
\address{First Lastname\\
Graduate School of Mathematics\\ XYZ University\\ City\\ Adresszusatz\\ ZIP\\ Country}
\curraddr{}
\email{first.last@university.edu}

\author[F.\ Last2]{First Last}
\address{First Lastname\\
Graduate School of Mathematics\\ XYZ University\\ City\\ Adresszusatz\\ ZIP\\ Country}
\curraddr{}
\email{first.last@university.edu}

\author[F.\ Last3]{First Last}
\address{First Lastname\\
	Graduate School of Mathematics\\ XYZ University\\ City\\ Adresszusatz\\ ZIP\\ Country}
\curraddr{}
\email{first.last@university.edu}

\author[F.\ Last4]{First Last}
\address{First Lastname\\
	Graduate School of Mathematics\\ XYZ University\\ City\\ Adresszusatz\\ ZIP\\ Country}
\curraddr{}
\email{first.last@university.edu}

\author[F.\ Last5]{First Last}
\address{First Lastname\\
	Graduate School of Mathematics\\ XYZ University\\ City\\ Adresszusatz\\ ZIP\\ Country}
\curraddr{}
\email{first.last@university.edu}

\subjclass[2010]{37P99}
\keywords{2-adic numbers, binary residue system}

\begin{document}
	
\begingroup
\let\MakeUppercase\relax
\maketitle
\endgroup

\begin{abstract}
This paper treats cycles in $kx+c$ functions. 
\end{abstract}

\newpage
{\renewcommand{\arraystretch}{1.8}
\begin{table}[H]
	\centering
	\begin{tabular}{|P{1.4cm} p{13.4cm}|}
		\hline
		\multicolumn{2}{|l|}{\thead[l]{\textbf{Fundamentals short and sweet}}}
		\\
		$C_{k,c}$ cycle & We consider the function $f_{k,c}(x)$ given by equation~\ref{eq:collatz_function}. A cycle $C_{k,c}$ is the sequence $(v_1, v_2,\ldots, v_n)$ of distinct positive integer, where $f_{k,c}(v_1)=v_2$ and $f_{k,c}(v_2)=v_3$ and so forth and finally $f_{k,c}(v_n)=v_1$.
		\\
		Primitive cycle & If all members of a cycle share a same common divisor greater than one, then this cycle is referred to as a \textit{non-primitve} cycle, otherwise it is a \textit{primitve} cycle.
		\\
		Non-reduced word & Let $C_{k,c}$ be a cycle having $U$ odd and $D$ even members. The non-reduced word describing this cycle is a word of length $U+D$ over the alphabet $\{u,d\}$, which has a $u$ at those positions, where an odd member and a $d$ where an even member is located in the cycle.
		\\
		Parity vector & Analoguously to the non-reduced word, the binary parity vector of a cycle $C_{k,c}=(v_1,v_2,\ldots,v_n)$ has $n=U+D$ entries. It has a $1$ at position $i$, if $v_i$ is odd, and otherwise $0$. For instance, we consider the non-reduced word $uuududuuuddudd$ synonymous to the parity vector $(1,1,1,0,1,0,1,1,1,0,0,1,0,0)$ or even simpler to the binary sequence (binary word) $11101011100100$.
		\\
		Interrelated cycles & Two cycles are called \textit{interrelated} if they have the same length and if they both have an equal amount of odd members, which means their non-reduced words contain an equal number of $u$ and $v$. Analoguously their parity vectors have the same number of zeros and ones. 
		\\ \hline
	\end{tabular}
\end{table}}

\section{Introduction}
\label{sec:introduction}

Starting point of our considerations is the function:
\begin{equation}
\label{eq:collatz_function}
f_{k,c}(x)=
\begin{cases}
\nicefrac{kx+c}{2}	&	2\nmid x\\
\nicefrac{x}{2}		&	\text{otherwise}
\end{cases}
\end{equation}

Let $S$ be a set containing two elements $u$ and $d$, which are bijective functions over $\mathbb{Q}$:
\begin{equation}
u(x)=\nicefrac{k\cdot x+c}{2}\hspace{4em} d(x)=\nicefrac{x}{2}
\end{equation}

Let a binary operation be the left-to-right composition of functions $u\circ d$, where $u\circ d(x)=d(u(x))$. $S^\ast$ is the monoid, which is freely generated by $S$. The identity element is the identity function $id_{\mathbb{Q}}=e$. We call $e$ an \textit{empty string}. $S^\ast$ consists of all expressions (strings) that can be concatenated from the generators $u$ and $d$. Every string can be written in precisely one way as product of factors $u$ and $d$ and natural exponents $e_i>0$:

\[
e,u^{e_1},d^{e_1},u^{e_1}d^{e_2},d^{e_1}u^{e_2},u^{e_1}d^{e_2}u^{e_3},d^{e_1}u^{e_2}d^{e_3},\ldots
\]

%https://de.wikipedia.org/wiki/Freie_Gruppe
These uniquely written products are called \textit{reduced words} over $S$. Using exponents $u_i,d_i>0$, we construct strings $s_i=u^{u_i}d^{d_i}$ and concatenate these to a larger string:

\[
s_1s_2\cdots s_n=u^{u_1}d^{d_1}u^{u_2}d^{d_2}\cdots u^{u_n}d^{d_n}
\]

Note that each string $s_i$ is a reduced word, since $u_i,d_i>0$. Let us evaluate this (large) string by inputting a natural number $v_1$. If the result is again $v_1$ then we obtain a cycle:

\[
u^{u_1}d^{d_1}u^{u_2}d^{d_2}\cdots u^{u_n}d^{d_n}(v_1)=d^{d_n}(u^{u_n}(\cdots d^{d_2}(u^{u_2}(d^{d_1}(u^{u_1}(v_1))))))=v_1
\]

We write the sums briefly as $U=u_1+\ldots+u_n$ and $D=d_1+\ldots+d_n$. The cycle contains $U+D$ elements. We summarize this fact to the following definition~\ref{def:odd_even_elements}:

\begin{definition}
\label{def:odd_even_elements}
A cycle consists of $U+D$ elements, where $U=u_1+\ldots+u_n$ is the number of its odd members and $D=d_1+\ldots+d_n$ the number of its even members.
\end{definition}

\par\noindent
Moreover we define $A=a_1+\ldots+a_n$ with

\[
a_i=2^{\sum_{j=1}^{i-1}u_j+d_j}\cdot\left(k^{u_i}-2^{u_i}\right)\cdot k^{\sum_{j=i+1}^{n}u_j}
\]

\par\medskip\noindent
Theorem~\ref{theo:v1} calculates the smallest member of the cycle $C_{k,c}$, which in line with definition~\ref{def:odd_even_elements} consists of $U$ odd and $D$ even members \cite{Ref_Gupta_2020}:

\begin{theorem}
\label{theo:v1}
The smallest number $v_1$ belonging to a cycle $C_{k,c}$ having $U$ odd and $D$ even members is:

\[
v_1=\frac{c\cdot A}{(k-2)(2^{U+D}-k^U)}
\]
\end{theorem}

\par\medskip
\begin{example}
\label{ex:C_3_11}
We consider a cycle $C_{3,11}$ that has $U+D=8+6=14$ elements and choose $(u_1,u_2,u_3,u_4)=(3,1,3,1)$ and $(d_1,d_2,d_3,d_4)=(1,1,2,2)$. Its smalles element is $v_1=13$ and we obtain all elements by evaluating the strings: $v_2=u(v_1)$, $v_3=u(v_2)$, $v_4=u(v_3)$ and $v_5=d(v_4)$ and so forth. It applies:
\[
uuud\circ ud\circ uuudd\circ udd(v_1)=u^3d\circ ud\circ u^3d^2\circ ud^2(v_1)=s_1\circ s_2\circ s_3\circ s_4(v_1)=v_1
\]

\par\noindent
This cycle is $(v_1,v_2,v_3,\ldots,v_{14})=(13,25,43,70,35,58,29,49,79,124,62,31,52,26)$. We calculate $v_1$ directly as follows:

\[
v_1=\frac{11\cdot 11609}{(3-2)(2^{8+6}-3^8)}=\frac{11\cdot11609}{9823}=13
\]

\par\noindent
In this case $11609=A=a_1+a_2+a_3+a_4=4617+1296+3648+2048$:

\[
\begin{array}{llll}
a_1=2^{0}&(3^3-2^3)&3^{1+3+1}&=4617\\
a_2=2^{3+1}&(3^1-2^1)&3^{3+1}&=1296\\
a_3=2^{3+1+1+1}&(3^3-2^3)&3^{1}&=3648\\
a_4=2^{3+1+1+1+3+2}&(3^1-2^1)&3^{0}&=2048
\end{array}
\]
\end{example}

\begin{theorem}
	\label{theo:max}
	The maximum odd element in cycle occurs immediately before the maximum even element.
\end{theorem}

\section{Conditions for cycles}
A positive integer $k$ is called a \textit{Crandall number}, if there exists a cycle $C_{k,1}$ and the following very fundamental theorem~\ref{theo:crandall_wieferich} is well known, see \cite{Ref_Crandall_1978}, \cite{Ref_Franco_Pomerance_1995}:

\begin{theorem}
\label{theo:crandall_wieferich}
Every Wieferich number is a Crandall number. In other words, if $k$ is a Wieferich number, then a cycle $C_{k,1}$ exists.
\end{theorem}

In conformity with definition~\ref{def:odd_even_elements}, let us consider a cycle $C_{k,c}=(v_1,v_2,\ldots,v_n)$ consisting of $U$ odd integers and $D$ even integers. We define a binary parity vector (it is synonymous to a binary sequence or binary non-reduced word) consisting of $n=U+D$ elements, which has a $1$ at position $i$, if $v_i$ is odd, and otherwise $0$. This vector corresponds to the non-reduced word over the alphabet $\{u,d\}$ as introduced in section~\ref{sec:introduction} having a $u$ at each position which in the vector is occupied by a $1$, and a $d$ at a position at which in the vector is a $0$. The theorem~\ref{theo:cycle_restriction_1} specifies several cycle restrictions:

\begin{theorem}
\label{theo:cycle_restriction_1}
For a cycle $C_{k,c}$ with $U$ odd and $D$ even members applies:
\par
\begin{enumerate}[label=(\alph*)]
\item A cycle only exists if the inequality $2^{U+D}-k^U>0$ holds.
\item The condition for the existence of a cycle can be detailed as follows \cite{Ref_Cox_2012}: A cycle $C_{k,c}$ only exists if the integer $c$ and the difference $2^{U+D}-k^U$ are not coprime: $\gcd(c,2^{U+D}-k^U)>1$.
\item Let $0\le x_1<x_2<\ldots<x_{U}<\le U-1$ be all positions (the indexing is zero-based) in the parity vector occupied by $1$ or equivalently all positions in the word $s\in S^\ast$ at which there is a letter $u$. A cycle $C_{k,c}$ only exists if the divisibility $2^{U+D}-k^U\mid c\cdot z(s)$ holds, where $z(s)=\sum_{i=1}^{U}=3^{U-i}2^{x_i}$.
\item The number of cycles $C_{k,c}$ is alsway less than or equal to the number of cycles $C_{k,a\cdot c}$ where $a$ is an odd number.
\end{enumerate}
\end{theorem}

\begin{example}
We refer to $C_{3,11}=(13,25,43,70,35,58,29,49,79,124,62,31,52,26)$ again. The corresponding parity vector is $(1,1,1,0,1,0,1,1,1,0,0,1,0,0)$ and the non-reduced word is $uuududuuuddudd$. The indices are $(x_1,\ldots,x_8)=(0,1,2,4,6,7,8,11)$ and therefore $z(8)=3^72^0+3^62^1+3^52^2+3^42^4+3^32^6+3^22^7+3^12^8+3^02^{11}=11609$.

\par\medskip\noindent
Correctly it applies that $2^{8+6}-3^8\mid11\cdot11609$, more specifically it is $9.823\mid127.699$ and $9.823\cdot13=127.699$. 
\end{example}

\begin{theorem}
\label{theo:cycle_uniqueness}
Two different primitive cycles $C_{k,c_1}$ and $C_{k,c_2}$ can never share a common parity vector.
\end{theorem}

\begin{proof}
We begin with the fact that a cycle $C_{k,c}$ with a given parity vector first appears at:

\[
c=\frac{(k-2)(2^{U+D}-k^{D})}{\gcd(A,(k-2)(2^{U+D}-k^{D}))}
\]

\par\medskip
Let there exist cycles $C_{k,c_1}$ and $C_{k,c_2}$ with the same parity vector, this implies that the values of $A$ and $(k-2)(2^{U+D}-k^{U})$ as defined in Definition \ref{def:odd_even_elements} are same for both the cycles. Therefore using the formula, a cycle can exist iff $v_1$ is an integer, i.e $c \cdot A$ divides $(k-2)(2^{U+D}-k^{U})$. The cycle will originate for the minimum such value of $c$. Therefore there can only be one value of $c$ for which the parity vector produces a cycle that is not inherited.
\end{proof}

%You can find out the value of M(l,n) and N(l,n) by running the Python program.  When you reduce a cycle, you have to divide M(l,n) and N(l,n) by the same amount to make comparisons with the maximum and minimum odd elements.  In the example cycle, l and n are not relatively prime.  In this case, this results in some cycles for c=47 (9823=11*19*47) having half the length and half the number of odd elements.  An example is 119, 202, 101, 175, 286, 143, 238.

\section{Boundary features of cycles}
Halbeisen and Hungerbühler \cite{Ref_Halbeisen_Hungerbuehler_1997} introduced a boundary feature as function $M(n,U)$, where $n$ is the cycle length and $U$ the number of odd members in that cycle. We will take this function as a basis for further considerations.

Let $n$ be the length of a cycle and $U$ the number of odd members in that cycle. Moreover, let $S_{n,U}$ denote the set of all binary words (sequences) of length $n$ containing exactly $U$ ones and otherwise only zeros. This set contains exactly $\binom{n}{U}$ words -- exactly the number of ways in which we may select $U$ elements out of $n$ total where the order is irrelevant. In the example given by table~\ref{table:calculation_M_5_2}, the elements of the set $S_{5,2}$ are all listed in the first column.

The left shift function $\lambda:S^\ast\rightarrow S^\ast$ rotates a binary word of length $n$ by one element to the left, for example $\lambda(uuddd)=udddu$ or $\lambda(11000)=10001$, see \cite{Ref_Halbeisen_Hungerbuehler_1997}. The second column of table~\ref{table:calculation_M_5_2} contains all words that result from the binary word $s\in S^\ast$ (given by the first column) left shifted up to $n$ times: $\lambda^1(s),\ldots,\lambda^5(s)$. In generalized terms, this set is denoted as $\sigma(s)=\{\lambda^i(s):1\le i\le n\}$.

The third column of table~\ref{table:calculation_M_5_2} contains the corresponding values $z(\lambda^1(s)),\ldots,z(\lambda^5(s))$ remembering that $z:S^\ast\rightarrow\mathbb{N}$ is the function, which we defined by theorem~\ref{theo:cycle_restriction_3}. The last column contains the minimum of these values. Finally, the largest of all these minima is $M(5,2)$ or generally, see \cite{Ref_Halbeisen_Hungerbuehler_1997}:

\begin{equation}
\label{eq:M_n_U}
M(n,U)=\max_{s\in S_{n,U}}\{\min_{t\in\sigma(s)}z(t)\}
\end{equation}

Additionally to Halbeisens and Hungerbühlers boundary feature $M(n,U)$ we introduce another boundary feature as a function $N(n,U)$. Let $r=\gcd(n, U)$, the function $N(n,U)$ is defined as follows:

\begin{equation}
\label{eq:N_n_U}
N(n,U)=2\cdot M(n,U)-\sum_{i=0}^{r-1}2^{i\cdot n/r}3^{U-1-i\cdot U/r}
\end{equation}

\begin{example}
\label{ex:M_N}
We choose a cycle $C_{k,c}=C_{3,23}$ of length $n=5$ having $U=2$ odd members, where $c=2^n-3^U=2^5-3^2=23$. Let us choose the binary words $11000$ and $10100$ and calculate the smallest member of the corresponding cycle in each case (using theorem~\ref{theo:v1}).

In the first case, namely $11000$ synonymous with $uuddd=u^2d^3=u^{u_1}d^{d_1}$ we obtain $v_1=\nicefrac{c\cdot A}{(k-2)(2^{U+D}-k^U)}=\nicefrac{23\cdot 5}{(3-2)(2^{2+3}-3^2)}=5$. The resulting cycle is $(5,19,40,20,10)$ which is given by the first row and third column in table~\ref{table:calculation_M_5_2}.

In the second case, $10100$ that is synonymous with $ududd=u^1d^1u^1d^2=u^{u_1}d^{d_1}u^{u_2}d^{d_2}$ we obtain $v_1=\nicefrac{23\cdot 7}{(3-2)(2^{2+3}-3^2)}=7$. The resulting cycle is $(7,22,11,28,14)$ which is given by the second row and third column in table~\ref{table:calculation_M_5_2}.

Table~\ref{table:calculation_M_5_2} exhibits how $M(n,U)$ is calculated, which in our concrete case is $M(5,2)=7$. Additionally we calculate $N(5,2)=2\cdot M(5,2)-2^03^{2-1-0}=14-3=11$.
\end{example}

\begin{table}[H]
	\centering
	\begin{tabular}{L|LLLL}
		\thead{} &
		\thead{\textbf{word }\boldsymbol{s}} &
		\thead{\textbf{set }\boldsymbol{\sigma(s)}\textbf{ of left shifted words}} &
		\thead{\boldsymbol{\{z(t):t\in\sigma(s)\}}} &
		\thead{\boldsymbol{\displaystyle \min_{t\in\sigma(s)}z(t)}}\\
		\hline
		\thead{\boldsymbol{1}} &
		11000 &
		11000,10001,00011,00110,01100 &
		5,19,40,20,10 &
		5
		\\
		\thead{\boldsymbol{2}} &		
		10100 &
		10100,01001,10010,00101,01010 &
		7,22,11,28,14 &
		7
		\\
		\thead{\boldsymbol{3}} &
		10010 &
		10010,00101,01010,10100,01001 &
		11,28,14,7,22 &
		7
		\\
		\thead{\boldsymbol{4}} &
		10001 &
		10001,00011,00110,01100,11000 &
		19,40,20,10,5 &
		5
		\\
		\thead{\boldsymbol{5}} &
		01100 &
		01100,11000,10001,00011,00110 &
		10,5,19,40,20 &
		5
		\\
		\thead{\boldsymbol{6}} &
		01010 &
		01010,10100,01001,10010,00101 &
		14,7,22,11,28 &
		7
		\\
		\thead{\boldsymbol{7}} &
		01001 &
		01001,10010,00101,01010,10100 &
		22,11,28,14,7 &
		7
		\\
		\thead{\boldsymbol{8}} &
		00110 &
		00110,01100,11000,10001,00011 &
		20,10,5,19,40 &
		5
		\\
		\thead{\boldsymbol{9}} &
		00101 &
		00101, 01010, 10100, 01001, 10010 &
		28,14,7,22,11 &
		7
		\\
		\thead{\boldsymbol{10}} &
		00011 &
		00011,00110,01100,11000,10001 &
		40,20,10,5,19 &
		5
		\\
		\hline
		\multicolumn{4}{r}{The largest of all minimum $z$ values is $M(n,U)=M(5,2)=$} &
		7
		\\
	\end{tabular}
	\caption{Calculation of $M(5,2)$}
	\label{table:calculation_M_5_2}
\end{table}

\section{Constructing one cycle from another}
Cycles may interrelate, which means they have the same length and an equal amount of odd members. We refer to example~\ref{ex:M_N} and consider the cycle $C_{3,23}=(5,19,40,20,10)$. A cycle, which interrelates to $C_{3,23}$ is for example $C_{3,69}=(15,57,120,60,30)$.

If we go back to example~\ref{ex:C_3_11}, then we can provide two interrelated cycles as well. For $k=3$ and $n=U+D=8+6=14$ we obtain $c=2^n-k^U=2^{14}-3^8=9823$ and the cycle $C_{3,9823}=(11609,22325,38399,62510,31255,51794,25897,43757,70547,110732,55366,\\27683,46436,23218)$.

When we divide the parameter $c$ and all cycle members by $893$, then we obtain the reduced interrelated cycle $C_{3,11}=(13,25,43,70,35,58,29,49,79,124,62,31,52,26)$.

\begin{theorem}
\label{theo:containment_M_N}
Let $C_{k,c}$ be a cycle of length $n=U+D$ that has $U$ odd and $D$ even members, where $c=2^n-k^U$. It always applies that $M(n,U)$ is greater than the smallest member and $N(n,U)$ is less than the largest odd member of this cycle.

If $c$ is divisable by an odd integer $a$, then for the (reduced) interrelated cycle $C_{k,\nicefrac{c}{a}}$ it applies that $\nicefrac{M(n,U)}{a}$ is greater than the smallest member and $\nicefrac{N(n,U)}{a}$ is less than the largest odd member of this reduced cycle.
\end{theorem}

\begin{example}
\label{ex:interrelated_M_N}
Starting point for us are again Cycles $C_{3,c}$. For $n=14$ and $U=8$ we have $c=2^{14}-3^8=9823$ and $M(14,8)=21109$ and $N(14,8)=36575$. The cycle's $C_{3,9823}$ smallest member is $11609$ and its largest odd member is $70547$. It applies that $11609\le M(14,8)$ and $N(14,8)\le70547$.

Since $c=9823=893\cdot11$ we are able to reduce the cycle $C_{3,9823}$ via division of $c$ and all its cycle members by $893$, which leads to the cycle $C_{3,11}$ having the smallest member $13$ and largest odd member $79$. The inequalities hold: $13\le\nicefrac{M(14,8)}{893}$ and $\nicefrac{N(14,8)}{893}\le79$, since $13\le23.638$ and $40.957\le79$.
\end{example}

\par
We use $C_{3,12885}$, hence $3x+12885$. Here $M(17,7)=39733$ and $N(17,7)=78737$. The sequence starting with $M(17,7)=39733$ is:

\[
\begin{array}{l}
39733,124042,62021,157474,78737,182548,91274,45637,132898,66449,164116,82058,\\
41029,125986,62993,158932,79466
\end{array}
\]

The parity vector for this sequence is $10101001010010100=86676$, which is the ceil vector (the biggest possible binary achievable by rotating the vector). When we rotate this vector by $r=17-12=5$ to the left or $r=12$ to the right, then we obtain the floor vector $00101001010010101=21141$ (the smallest possible binary achievable by rotsting the vector).

Conversly, when we rotate the floor vector by $r=5$ to the right or $r=12$ to the left, then we obtain the ceil vector.

Let $x$ be an integer, $l$ the length of the vector and $r$ the number of circular left shifts (left rotations). It gerally applies:

\[
\begin{array}{l}
rotateLeft(x,r,l)=(x\cdot2^r)\bmod{(2^l-1)}\\
rotateRight(x,r,l)=rotateLeft(x,l-r,l)
\end{array}
\]

Therefore $rotateLeft(21141,12,17)\equiv(21141\cdot2^{12})\pmod{(2^{17}-1)}$.
\par\noindent
And $rotateLeft(21141,12,17)=21141\cdot2^{12}\bmod{(2^{17}-1)}=86676$. Respectively $86593536\bmod{131071}=86676$.

We need to show that $l\mid n\cdot r+1$, which in our case is also true since $17\mid 7\cdot12+1$. The value $r=12$ can be considered as the rotation distance between the largest and smallest possible binary in that partity vector.

\vspace{1em}
\bibliographystyle{unsrt}
\bibliography{references}

\end{document}