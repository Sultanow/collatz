\documentclass[12pt]{amsart}
\usepackage{enumerate}
\usepackage[colorlinks=true, linkcolor=blue, urlcolor=blue, citecolor=blue, anchorcolor=blue, pdfborder={0 0 0}]{hyperref}
\usepackage{url}
\usepackage{graphicx,color}
\usepackage{cite}
\usepackage{amsthm, amsmath, amssymb}
\usepackage{mathtools}
\usepackage[top=45truemm, bottom=45truemm, left=30truemm, right=30truemm]{geometry}
\usepackage{nicefrac}
\usepackage{cancel}
\usepackage{float}
\usepackage{tabularx}
\usepackage{makecell}
\usepackage{array}
\usepackage{ragged2e}

\newcolumntype{P}[1]{>{\RaggedRight\hspace{0pt}}p{#1}}

\newcolumntype{L}{>{\begin{math}}l<{\end{math}}}%
\newcolumntype{C}{>{\begin{math}}c<{\end{math}}}%
\newcolumntype{R}{>{\begin{math}}r<{\end{math}}}%

\newtheorem{theorem}{Theorem}[section]
\newtheorem{lemma}[theorem]{Lemma}
\newtheorem{corollary}[theorem]{Corollary}
\newtheorem{definition}[theorem]{Definition}
\newtheorem{proposition}[theorem]{Proposition}
\newtheorem{example}[theorem]{Example}
\theoremstyle{definition}
\newtheorem{remark}[theorem]{Remark}

\setlength{\headsep}{2em}
\setlength{\skip\footins}{1.4pc plus 5pt minus 2pt}

\title[Cycles in kx+c functions]{Cycles in $kx+c$ functions}

\author[F.\ Last1]{\href{https://orcid.org/0000-0000-0000-0000}{\includegraphics[scale=0.06]{orcid.png}}\hspace{1mm}First Last}
\address{First Lastname\\
Graduate School of Mathematics\\ XYZ University\\ City\\ Adresszusatz\\ ZIP\\ Country}
\curraddr{}
\email{first.last@university.edu}

\author[F.\ Last2]{First Last}
\address{First Lastname\\
Graduate School of Mathematics\\ XYZ University\\ City\\ Adresszusatz\\ ZIP\\ Country}
\curraddr{}
\email{first.last@university.edu}

\author[F.\ Last3]{First Last}
\address{First Lastname\\
	Graduate School of Mathematics\\ XYZ University\\ City\\ Adresszusatz\\ ZIP\\ Country}
\curraddr{}
\email{first.last@university.edu}

\author[F.\ Last4]{First Last}
\address{First Lastname\\
	Graduate School of Mathematics\\ XYZ University\\ City\\ Adresszusatz\\ ZIP\\ Country}
\curraddr{}
\email{first.last@university.edu}

\author[F.\ Last5]{First Last}
\address{First Lastname\\
	Graduate School of Mathematics\\ XYZ University\\ City\\ Adresszusatz\\ ZIP\\ Country}
\curraddr{}
\email{first.last@university.edu}

\subjclass[2010]{37P99}
\keywords{2-adic numbers, binary residue system}

\begin{document}
	
\begingroup
\let\MakeUppercase\relax
\maketitle
\endgroup

\begin{abstract}
This paper treats cycles in $kx+c$ functions. 
\end{abstract}

\newpage
{\renewcommand{\arraystretch}{1.8}
\begin{table}[H]
	\centering
	\begin{tabular}{|P{1.4cm} p{13.4cm}|}
		\hline
		\multicolumn{2}{|l|}{\thead[l]{\textbf{Fundamentals short and sweet}}}
		\\
		cycle & A cycle is.
		\\
		dead limb & A dead limb is.
		\\
		unitary ring & A unitary ring is a ring with a multiplicative identity $1$ (which differs from the additive identity $1\ne0$) such that $1a=a=a1$ for all elements $a$ of the ring.
		\\
		Ideal & Let $(R,+,\cdot)$ be a commutative unitary ring. Then the subset $I\subseteq R$ is called an ideal of $R$ if $(I,+)$ is a commutative group and if $xI\subseteq I$ for all $x\in R$, see \cite[p.~66-67]{Ref_Wolfart_2011}.
		\\
		quot. ring & Using an ideal of a ring $I\subseteq R$, we may define an equivalence relation $\sim$ on $R$ by $a\sim b$ iff $a-b$ is in $I$ \cite[p.~69]{Ref_Schulze-Pillot_2015}. The equivalence class of $a$ in $R$ is given by $[a]=a+I:=\{a+r|r\in I\}$ for $r\in R$ and referred to as "residue class of $a$ modulo $I$", see \cite[p.~122]{Ref_Schubert_2012}, \cite[p.~70]{Ref_Schulze-Pillot_2015}. The set of all these equivalence classes becomes the quotient ring (residue class ring) modulo the ideal $I$, denoted by $R/I$.
		\\
		compl. residue system & Let $I\subseteq R$ be an ideal and $[a]$ the residue classes of $a$ modulo $I$, which means that $a+I=b+I$ when $a\equiv b\mod I$ or respectively $a-b\in I$ \cite[p.~70]{Ref_Schulze-Pillot_2015}. $R$ is the disjoint union of the different residue classes $a$ modulo $I$. A subset $M\subseteq R$, which contains exactly one element from each of these residue classes, is called a complete residue system of $R$ modulo $I$, see \cite[p.~70]{Ref_Schulze-Pillot_2015}.
		\\
		$[a]_n$ & The residue class (also termed congruence class) of the integers for a modulus $n$ is the set $[a]_n=\{a+kn|k\in\mathbb{Z}\}$ and sometimes denoted by $\bar a_n$ or by $a+n\mathbb{Z}$, see \cite[p.~15]{Ref_Wolfart_2011}, \cite[p.~122]{Ref_Schubert_2012}, \cite[p.~25]{Ref_Mueller-Stach_2011}.
		\\
		$\mathbb{Z}/n\mathbb{Z}$ & The set of all residue classes $[a]_n$ is called the ring of integers modulo $n$ and denoted by $\mathbb{Z}/n\mathbb{Z}=\{[a]_n|a\in\mathbb{Z}\}$ and trivially $\mathbb{Z}/0\mathbb{Z}=\mathbb{Z}$ and for all $n\ne0$ we have $\mathbb{Z}/n\mathbb{Z}=\{[0],[1],\ldots,[n-1]\}$, see \cite[p.~15]{Ref_Wolfart_2011}, \cite[p.~25]{Ref_Mueller-Stach_2011}. 
		\\ \hline
	\end{tabular}
\end{table}}

{\renewcommand{\arraystretch}{1.8}
\begin{table}[H]
	\centering
	\begin{tabular}{|P{1.4cm} p{13.4cm}|}
		\hline
		direct prod. & If $R_1,R_2,\ldots,R_n$ are rings, the cartesian product $R_1\times R_2\times\ldots\times R_n$ forms the set of all ordered $n$-tuples $(r_1,r_2,\ldots,r_n)$, where $r_i\in R_i$. The addition and multiplication of these n-tuples is defined "coordinatewise" by components. The resulting ring is called a "direct product" of the original rings $R_i$ \cite[p.~51]{Ref_Wolfart_2011}, \cite[p.~169]{Ref_Fraleigh_2014}.
		\\
		prod. of ideals & Let $I,J$ be two ideals of a ring $R$. Their product $IJ$ is defined as the set of all finite sums $a_1b_1+\ldots+a_nb_n$ in which $n\ge0$ and $a_1,\ldots,a_n\in I$ and $b_1,\ldots,b_n\in J$, see \cite[p.~87]{Ref_Schmidt_2007}.
		\\
		power of an ideal & Let $I$ be an ideal in $R$. The $n$-th power of $I$, denoted by $I^n$, is the n-times product of the ideal $I$ with itself. $I^n$ contains sums of elements of the form $a_1a_2\cdots a_n$ where $a_1,a_2,\ldots,a_n\in I$ and the products refer to the multiplication defined in $R$. Consequently, $I^{n+1}\subseteq I^n$. Note that the product and power of ideals should not be confused with the direct product of rings.
		\\
		filtration & Let $R$ be a ring. A sequence of ideals $I_0,I_1,I_2,\ldots$ is said to be a "filtration" on $R$ if $I_0=R$ and for each integer $j\ge0$ applies that $I_j\supseteq I_{j+1}$ and if $I_jI_k\subseteq I_{j+k}$, see \cite[p.~269]{Ref_Lucas_2001}.
		\\
		princip. ideal & A "principle ideal" is an ideal in a ring $R$ which is generated by a single element $a$ of $R$ through multiplication by every element of $R$. There are some rings in which every ideal is a principle ideal, so-called "principle ideal rings" \cite[p.~68]{Ref_Wolfart_2011}.
		\\
		max. ideal & A proper Ideal $M$ of a ring $R$ is called "maximal ideal" of $R$  if there is no other proper ideal $N$ of $R$ properly containing $M$ \cite[p.~247]{Ref_Fraleigh_2014}, \cite[p.~37]{Ref_Northcott_1953}. A Note on "proper containment": If $R$ is any set, then $R$ is the improper subset of $R$. Any other subset $N\ne R$ is a proper subset of $R$ and denoted by $N\subset R$ or $N\varsubsetneq R$ \cite[p.~2]{Ref_Fraleigh_2014}.
		\\
		prime ideal & Let $a$ and $b$ are two elements of $R$ and $P$ a proper ideal such that their product $ab$ is an element of $P$. $P$ is called a prime ideal if at least one of $a$ and $b$ belongs to $P$, in other words from $ab\in P$ and $a\notin P$ always follows $b\in P$ \cite[p.~9]{Ref_Northcott_1953}.
		\\
		max. prime ideal & A proper prime ideal $P$ is said to be a "maximal prime ideal" of the ring $R$, if there is no other proper prime ideal containing $P$ \cite[p.~23]{Ref_Northcott_1953}.
		\\
		local ring & A commutative ring $R$ is called a local ring if it has a unique maximal ideal $M$ \cite[p.~522]{Ref_Rotman_2005}.
		\\ \hline
	\end{tabular}
\end{table}}

\section{Introduction}
\label{introduction}

Let $S$ be a set containing two elements $u$ and $d$, which are bijective functions over $\mathbb{Q}$:
\begin{equation}
u(x)=\nicefrac{k\cdot x+c}{2}\hspace{4em} d(x)=\nicefrac{x}{2}
\end{equation}

Let a binary operation be the left-to-right composition of functions $u\circ d$, where $u\circ d(x)=d(u(x))$. $S^\ast$ is the monoid, which is freely generated by $S$. The identity element is the identity function $id_{\mathbb{Q}}=e$. We call $e$ an \textit{empty string}. $S*$ consists of all expressions (strings) that can be concatenated from the generators $u$ and $d$. Every string can be written in precisely one way as product of factors $u$ and $d$ and natural exponents $e_i>0$:

\[
e,u^{e_1},d^{e_1},u^{e_1}d^{e_2},d^{e_1}u^{e_2},u^{e_1}d^{e_2}u^{e_3},d^{e_1}u^{e_2}d^{e_3},\ldots
\]

%https://de.wikipedia.org/wiki/Freie_Gruppe
These uniquely written products are called \textit{reduced words} over $S$. We construct strings $s_i=u^{u_i}d^{d_i}$ which we concatenate to a larger string:

\[
s_1s_2\cdots s_n=u^{u_1}d^{d_1}u^{u_2}d^{d_2}\cdots u^{u_n}d^{d_n}
\]

Let us evaluate this (large) string by inputting a natural number $v_0$. If the result is again $v_0$ then we obtain a cycle:

\[
u^{u_1}d^{d_1}u^{u_2}d^{d_2}\cdots u^{u_n}d^{d_n}(v_0)=d^{d_n}(u^{u_n}(\cdots d^{d_2}(u^{u_2}(d^{d_1}(u^{u_1}(v_0))))))=v_0
\]

We write the sums briefly as $U=u_1+\ldots+u_n$ and $D=d_1+\ldots+d_n$. The cycle contains $U+D$ elements. Moreover we define $A=a_1+\ldots+a_n$ with

\[
a_i=2^{\sum_{j=1}^{i-1}u_j+d_j}\cdot\left(k^{u_i}-2^{u_i}\right)\cdot k^{\sum_{j=i+1}^{n}u_j}
\]

The smallest number $v_0$ belonging to this cycle can be calculated by the following equation~\ref{eq:cycle_v0}:

\begin{equation}
\label{eq:cycle_v0}
v_0=\frac{c\cdot A}{(k-2)(2^{U+D}-k^U)}
\end{equation}

\begin{example}
For $k=3$, $c=11$ we choose $(u_1,u_2,u_3,u_4)=(3,1,3,1)$ and $(d_1,d_2,d_3,d_4)=(1,1,2,2)$. Therefore we consider a cycle for $3x+11$, which has $U+D=8+6=14$ elements. Its smalles element is $v_0=13$ and we obtain all elements by evaluating the strings: $v_1=u(v_0)$, $v_2=u(v_1)$, $v_3=u(v_2)$ and $v_4=d(v_3)$ and so forth. It applies:

\[
uuud\circ ud\circ uuudd\circ udd(v_0)=u^3d\circ ud\circ u^3d^2\circ ud^2(v_0)=s_1\circ s_2\circ s_3\circ s_4(v_0)=v_0
\]

\par\medskip\noindent
This cycle is $(v_0,v_1,v_2,\ldots,v_{14})=(13,25,43,70,35,58,29,49,79,124,62,31,52,26)$. We calculate $v_0$ directly as follows:

\[
v_0=\frac{11\cdot 11609}{(3-2)(2^{8+6}-3^8)}=\frac{11\cdot11609}{9823}=13
\]

\par\noindent
In this case $11609=A=a_1+a_2+a_3+a_4=4617+1296+3648+2048$:

\[
\begin{array}{llll}
a_1=2^{0}&(3^3-2^3)&3^{1+3+1}&=4617\\
a_2=2^{3+1}&(3^1-2^1)&3^{3+1}&=1296\\
a_3=2^{3+1+1+1}&(3^3-2^3)&3^{1}&=3648\\
a_4=2^{3+1+1+1+3+2}&(3^1-2^1)&3^{0}&=2048
\end{array}
\]
\end{example}

\section{Conditions under which a cycle can occur}

Anant and Darrel: Please let us collect all known restrictions under which a cycle can occur, by referencing all that is well-known and referencable and by deriving conditions from all we have.

The following restrictions exists for cycles. A cycle for $kx+c$ only exists if the inequality~\ref{eq:cycle_restriction_1} holds:

\begin{equation}
\label{eq:cycle_restriction_1}
2^{U+D}-k^U>0
\end{equation}

Another restriction given by theorem~\ref{theo:cycle_restriction_2}:

\begin{theorem}
\label{theo:cycle_restriction_2}
The number of cycles for $kx+c$ is alsway less than or equal to the number of cycles for $kx+ac$ where $a$ is an odd number.
\end{theorem}

\vspace{1em}
\bibliographystyle{unsrt}
\bibliography{references}

\end{document}