\documentclass[12pt]{amsart}
\usepackage{enumerate}
\usepackage[colorlinks=true, linkcolor=blue, urlcolor=blue, citecolor=blue, anchorcolor=blue, pdfborder={0 0 0}]{hyperref}
\usepackage{url}
\usepackage{graphicx,color}
\usepackage{cite}
\usepackage{amsthm, amsmath, amssymb}
\usepackage{mathtools}
\usepackage[top=45truemm, bottom=45truemm, left=30truemm, right=30truemm]{geometry}
\usepackage{nicefrac}
\usepackage{cancel}
\usepackage{float}
\usepackage{tabularx}
\usepackage{makecell}
\usepackage{array}
\usepackage{ragged2e}

\newcolumntype{P}[1]{>{\RaggedRight\hspace{0pt}}p{#1}}

\newcolumntype{L}{>{\begin{math}}l<{\end{math}}}%
\newcolumntype{C}{>{\begin{math}}c<{\end{math}}}%
\newcolumntype{R}{>{\begin{math}}r<{\end{math}}}%

\newtheorem{theorem}{Theorem}[section]
\newtheorem{lemma}[theorem]{Lemma}
\newtheorem{corollary}[theorem]{Corollary}
\newtheorem{definition}[theorem]{Definition}
\newtheorem{proposition}[theorem]{Proposition}
\newtheorem{example}[theorem]{Example}
\theoremstyle{definition}
\newtheorem{remark}[theorem]{Remark}

\setlength{\headsep}{2em}
\setlength{\skip\footins}{1.4pc plus 5pt minus 2pt}

\title[Cycles in kx+c functions]{Cycles in $kx+c$ functions}

\author[F.\ Last1]{\href{https://orcid.org/0000-0000-0000-0000}{\includegraphics[scale=0.06]{orcid.png}}\hspace{1mm}First Last}
\address{First Lastname\\
Graduate School of Mathematics\\ XYZ University\\ City\\ Adresszusatz\\ ZIP\\ Country}
\curraddr{}
\email{first.last@university.edu}

\author[F.\ Last2]{First Last}
\address{First Lastname\\
Graduate School of Mathematics\\ XYZ University\\ City\\ Adresszusatz\\ ZIP\\ Country}
\curraddr{}
\email{first.last@university.edu}

\author[F.\ Last3]{First Last}
\address{First Lastname\\
	Graduate School of Mathematics\\ XYZ University\\ City\\ Adresszusatz\\ ZIP\\ Country}
\curraddr{}
\email{first.last@university.edu}

\author[F.\ Last4]{First Last}
\address{First Lastname\\
	Graduate School of Mathematics\\ XYZ University\\ City\\ Adresszusatz\\ ZIP\\ Country}
\curraddr{}
\email{first.last@university.edu}

\author[F.\ Last5]{First Last}
\address{First Lastname\\
	Graduate School of Mathematics\\ XYZ University\\ City\\ Adresszusatz\\ ZIP\\ Country}
\curraddr{}
\email{first.last@university.edu}

\subjclass[2010]{37P99}
\keywords{2-adic numbers, binary residue system}

\begin{document}
	
\begingroup
\let\MakeUppercase\relax
\maketitle
\endgroup

\begin{abstract}
This paper treats cycles in $kx+c$ functions. 
\end{abstract}

\newpage
{\renewcommand{\arraystretch}{1.8}
\begin{table}[H]
	\centering
	\begin{tabular}{|P{1.4cm} p{13.4cm}|}
		\hline
		\multicolumn{2}{|l|}{\thead[l]{\textbf{Fundamentals short and sweet}}}
		\\
		$C_{k,c}$ cycle & We consider the function $f_{k,c}(x)$ given by equation~\ref{eq:collatz_function}. A cycle $C_{k,c}$ is the sequence $(v_1, v_2,\ldots, v_n)$ of distinct positive integer, where $f_{k,c}(v_1)=v_2$ and $f_{k,c}(v_2)=v_3$ and so forth and finally $f_{k,c}(v_n)=v_1$.
		\\
		Primitive cycle & If all members of a cycle share a same common divisor greater than one, then this cycle is referred to as a \textit{non-primitve} cycle, otherwise it is a \textit{primitve} cycle.
		\\
		Non-reduced word & Let $C_{k,c}$ be a cycle having $U$ odd and $D$ even members. The non-reduced word describing this cycle is a word of length $U+D$ over the alphabet $\{u,d\}$, which has a $u$ at those positions, where an odd member and a $d$ where an even member is located in the cycle.
		\\
		Parity vector & Analoguously to the non-reduced word, the (binary) parity vector of a cycle $C_{k,c}=(v_1,v_2,\ldots,v_n)$ has $n=U+D$ entries. It has a $1$ at position $i$, if $v_i$ is odd, and otherwise $0$.
		\\
		Interrelated cycles & Two cycles are called \textit{interrelated} if they have the same length and if they both have an equal amount of odd members, which means their non-reduced words contain an equal number of $u$ and $v$. Analoguously their parity vectors have the same number of zeros and ones.
		\\
		dead limb & A dead limb is. 
		\\ \hline
	\end{tabular}
\end{table}}

\section{Introduction}
\label{sec:introduction}

Starting point of our considerations is the function:
\begin{equation}
\label{eq:collatz_function}
f_{k,c}(x)=
\begin{cases}
\nicefrac{kx+c}{2}	&	2\nmid x\\
\nicefrac{x}{2}		&	\text{otherwise}
\end{cases}
\end{equation}

Let $S$ be a set containing two elements $u$ and $d$, which are bijective functions over $\mathbb{Q}$:
\begin{equation}
u(x)=\nicefrac{k\cdot x+c}{2}\hspace{4em} d(x)=\nicefrac{x}{2}
\end{equation}

Let a binary operation be the left-to-right composition of functions $u\circ d$, where $u\circ d(x)=d(u(x))$. $S^\ast$ is the monoid, which is freely generated by $S$. The identity element is the identity function $id_{\mathbb{Q}}=e$. We call $e$ an \textit{empty string}. $S*$ consists of all expressions (strings) that can be concatenated from the generators $u$ and $d$. Every string can be written in precisely one way as product of factors $u$ and $d$ and natural exponents $e_i>0$:

\[
e,u^{e_1},d^{e_1},u^{e_1}d^{e_2},d^{e_1}u^{e_2},u^{e_1}d^{e_2}u^{e_3},d^{e_1}u^{e_2}d^{e_3},\ldots
\]

%https://de.wikipedia.org/wiki/Freie_Gruppe
These uniquely written products are called \textit{reduced words} over $S$. Using exponents $u_i,d_i>0$, we construct strings $s_i=u^{u_i}d^{d_i}$ and concatenate these to a larger string:

\[
s_1s_2\cdots s_n=u^{u_1}d^{d_1}u^{u_2}d^{d_2}\cdots u^{u_n}d^{d_n}
\]

Note that each string $s_i$ is a reduced word, since $u_i,d_i>0$. Let us evaluate this (large) string by inputting a natural number $v_1$. If the result is again $v_1$ then we obtain a cycle:

\[
u^{u_1}d^{d_1}u^{u_2}d^{d_2}\cdots u^{u_n}d^{d_n}(v_1)=d^{d_n}(u^{u_n}(\cdots d^{d_2}(u^{u_2}(d^{d_1}(u^{u_1}(v_1))))))=v_1
\]

We write the sums briefly as $U=u_1+\ldots+u_n$ and $D=d_1+\ldots+d_n$. The cycle contains $U+D$ elements. We summarize this fact to the following definition~\ref{def:odd_even_elements}:

\begin{definition}
\label{def:odd_even_elements}
A cycle consists of $U+D$ elements, where $U=u_1+\ldots+u_n$ is the number of its odd members and $D=d_1+\ldots+d_n$ the number of its even members.
\end{definition}

\par\noindent
Moreover we define $A=a_1+\ldots+a_n$ with

\[
a_i=2^{\sum_{j=1}^{i-1}u_j+d_j}\cdot\left(k^{u_i}-2^{u_i}\right)\cdot k^{\sum_{j=i+1}^{n}u_j}
\]

\par\medskip\noindent
Theorem~\ref{theo:v0} calculates the smallest member of the cycle $C_{k,c}$, which in line with definition~\ref{def:odd_even_elements} consists of $U$ odd and $D$ even members \cite{Ref_Gupta_2020}:

\begin{theorem}
\label{theo:v0}
The smallest number $v_1$ belonging to a cycle $C_{k,c}$ having $U$ odd and $D$ even members is:

\[
v_1=\frac{c\cdot A}{(k-2)(2^{U+D}-k^U)}
\]
\end{theorem}

\par\medskip
\begin{example}
We consider a cycle $C_{3,11}$ that has $U+D=8+6=14$ elements and choose $(u_1,u_2,u_3,u_4)=(3,1,3,1)$ and $(d_1,d_2,d_3,d_4)=(1,1,2,2)$. Its smalles element is $v_1=13$ and we obtain all elements by evaluating the strings: $v_2=u(v_1)$, $v_3=u(v_2)$, $v_4=u(v_3)$ and $v_5=d(v_4)$ and so forth. It applies:
\[
uuud\circ ud\circ uuudd\circ udd(v_1)=u^3d\circ ud\circ u^3d^2\circ ud^2(v_1)=s_1\circ s_2\circ s_3\circ s_4(v_1)=v_1
\]

\par\noindent
This cycle is $(v_1,v_2,v_3,\ldots,v_{14})=(13,25,43,70,35,58,29,49,79,124,62,31,52,26)$. We calculate $v_1$ directly as follows:

\[
v_1=\frac{11\cdot 11609}{(3-2)(2^{8+6}-3^8)}=\frac{11\cdot11609}{9823}=13
\]

\par\noindent
In this case $11609=A=a_1+a_2+a_3+a_4=4617+1296+3648+2048$:

\[
\begin{array}{llll}
a_1=2^{0}&(3^3-2^3)&3^{1+3+1}&=4617\\
a_2=2^{3+1}&(3^1-2^1)&3^{3+1}&=1296\\
a_3=2^{3+1+1+1}&(3^3-2^3)&3^{1}&=3648\\
a_4=2^{3+1+1+1+3+2}&(3^1-2^1)&3^{0}&=2048
\end{array}
\]
\end{example}

\section{Conditions for cycles}
A positive integer $k$ is called a \textit{Crandall number}, if there exists a cycle $C_{k,1}$ and the following very fundamental theorem~\ref{theo:crandall_wieferich} is well known, see \cite{Ref_Crandall_1978}, \cite{Ref_Franco_Pomerance_1995}:

\begin{theorem}
\label{theo:crandall_wieferich}
Every Wieferich number is a Crandall number. In other words, if $k$ is a Wieferich number, then a cycle $C_{k,1}$ exists.
\end{theorem}

In conformity with definition~\ref{def:odd_even_elements}, let us consider a cycle $C_{k,c}$ consisting of $U$ odd integers and $D$ even integers. The theorem~\ref{theo:cycle_restriction_1} specifies the following cycle restriction:

\begin{theorem}
\label{theo:cycle_restriction_1}
A cycle only exists if the inequality $2^{U+D}-k^U>0$ holds.
\end{theorem}

The following theorem details the condition for the existence of a cycle \cite{Ref_Cox_2012}:

\begin{theorem}
\label{theo:cycle_restriction_2}
A cycle $C_{k,c}$ only exists if $c\mid2^{U+D}-k^U$.
\end{theorem}

For a sequence $(v_1,v_2,\ldots,v_n)$ of numbers, let us define a (binary) parity vector consisting of $n$ elements, which has a $1$ at position $i$, if $v_i$ is odd, and otherwise $0$. This vector corresponds to the non-reduced word over the alphabet $\{u,d\}$ as introduced in section~\ref{sec:introduction}. This word has a $u$ at each position which in the vector is occupied by a $1$, and it has a $v$ at a position at which in the vector is a $0$.

Let $0\le x_1<x_2<\ldots<x_{U}<\le U-1$ be all positions (the indexing is zero-based) in the parity vector occupied by $1$ or equivalently all positions in the word at which there is a $u$. We can detail theorem~\ref{theo:cycle_restriction_2} as follows by theorem~\ref{theo:cycle_restriction_3}:

\begin{theorem}
\label{theo:cycle_restriction_3}
A cycle $C_{k,c}$ only exists if the divisibility $2^{U+D}-k^U\mid c\cdot z$ holds, where $z=\sum_{i=1}^{U}=3^{U-i}2^{x_i}$.
\end{theorem}

\begin{example}
We refer to $C_{3,11}=(13,25,43,70,35,58,29,49,79,124,62,31,52,26)$ again. The corresponding parity vector is $(1,1,1,0,1,0,1,1,1,0,0,1,0,0)$ and the non-reduced word is $uuududuuuddudd$. The indices are $(x_1,\ldots,x_8)=(0,1,2,4,6,7,8,11)$ and therefore $z=3^72^0+3^62^1+3^52^2+3^42^4+3^32^6+3^22^7+3^12^8+3^02^{11}=11609$.

\par\medskip\noindent
Correctly it applies that $2^{8+6}-3^8\mid11\cdot11609$, more specifically it is $9.823\mid127.699$ and $9.823\cdot13=127.699$. 
\end{example}

%You can find out the value of M(l,n) and N(l,n) by running the Python program.  When you reduce a cycle, you have to divide M(l,n) and N(l,n) by the same amount to make comparisons with the maximum and minimum odd elements.  In the example cycle, l and n are not relatively prime.  In this case, this results in some cycles for c=47 (9823=11*19*47) having half the length and half the number of odd elements.  An example is 119, 202, 101, 175, 286, 143, 238.

\par\noindent
Another restriction given by theorem~\ref{theo:cycle_restriction_4}:

\begin{theorem}
	\label{theo:cycle_restriction_4}
	The number of cycles $C_{k,c}$ is alsway less than or equal to the number of cycles $C_{k,a\cdot c}$ where $a$ is an odd number.
\end{theorem}

\section{Constructing one cycle from another}




\vspace{1em}
\bibliographystyle{unsrt}
\bibliography{references}

\end{document}